\documentclass[11pt,a4paper]{article}

\usepackage[utf8]{inputenc}
\usepackage[T1]{fontenc}
\usepackage[english]{babel}
\usepackage[pdftex]{graphicx} 
\usepackage{amsmath}
\usepackage{amssymb}
\usepackage{amsthm}
\usepackage{mathabx}
\usepackage{indentfirst}
\usepackage{setspace}
\usepackage[usenames,svgnames,dvipsnames]{xcolor}
\usepackage[fixlanguage]{babelbib}
\usepackage[round,sort,nonamebreak]{natbib} 

\usepackage[pdftex,plainpages=false,pdfpagelabels,colorlinks=true,citecolor=Black,linkcolor=NavyBlue,urlcolor=DarkRed,filecolor=green,bookmarksopen=true]{hyperref}
\usepackage[all]{hypcap} 

\graphicspath{{../figuras/}}
\frenchspacing
\urlstyle{same}
\raggedbottom
%\fontsize{60}{62}\usefont{OT1}{cmr}{m}{n}{\selectfont}
%\cleardoublepage
\normalsize

\newcommand{\CC}{\mathcal{C}}
\newcommand{\LL}{\mathcal{L}}
\newcommand{\MM}{\mathcal{M}}
\newcommand{\PP}{\mathcal{P}}
\newcommand{\TT}{\mathcal{T}}
\newcommand{\RR}{\mathcal{R}}

\newcommand{\N}{{\mathbb{N}}}
\newcommand{\Nb}{{\widebar{\N}}}
\newcommand{\Nz}{{\mathbb{N^*}}}
\newcommand{\Nzb}{{\mathbb{\widebar{N}^*}}}
\newcommand{\Z}{{\mathbb{Z}}}
\newcommand{\R}{{\mathbb{R}}}
\newcommand{\E}{{\mathbb{E}}}
\newcommand{\ind}{{\mathbb{I}}}
\newcommand{\diam}{{\mathrm{diam}}}
\newcommand{\var}{\mathop{\mathrm{Var}}}

\newtheorem{teorema}{Theorem}[section]
\newtheorem{lema}[teorema]{Lemma}
\newtheorem{proposicao}[teorema]{Proposition}
\newtheorem{corolario}[teorema]{Corollary}


\title{A study on the non-homogeneous K-Process}
\author{
  Gabriel R. C. Peixoto
  \and
  Luiz Renato G. Fontes
}
\date{\today}

\begin{document}

\maketitle


% ---------------------------------------------------------------------------- %
% Abstract
\section*{Abstract}

We generalize the construction of K-Process given in \cite{fontes:08},
to contemplate states with non-homogeneous entrance rates. This type
of model was first introduced by \cite{kolmogorov:51} as a source of
counter examples. It is now know that they are a scaling limit for
Trap Models, which led attention back to them.

We proved the strong markov propriety of our construction, calculated
the transition rates and the infinitesimal generator.


% ---------------------------------------------------------------------------- %
% Bibliografia
\singlespacing
\bibliographystyle{plainnat-ime}
\bibliography{../bibliografia}

\end{document}


%%% Local Variables: 
%%% mode: latex
%%% TeX-master: 
%%% End: 
