% ---------------------------------------------------------------------------- %
% Agradecimentos
\chapter*{Agradecimentos}

Primeiramente gostaria de agradecer aos meus pais, José Milton e Hebe,
que sempre priorizaram minha educação, me deram todas as oportunidades
que eles puderam e me apoiaram em todas as minhas empreitadas.

Agradeço meu orientador, Prof. Dr. Luiz Renato Fontes, que sempre
teve paciência de discutir e me guiar quando estive perdido durante
esse trabalho, mas sempre teve o cuidado de não fazer o trabalho por
mim. Aprendi muito nesse processo.

Agradeço a minha família, que me aturam até hoje. As baixinhas Raquel,
Beatriz e Fernanda e meus avós Milton, Neide e
Zina. %ia escrever enedina e ambrozina, mas deixa quieto... :P

Agradeço à todos meus professores, cujos ensinamentos me trouxeram até
aqui.  Em especial ao Prof. Dr. Pablo Augusto Ferrari, que me orientou
durante minha iniciação científica, e me fez amadurecer em muito minha
visão da matemática.

Agradeço à todos meus amigos, sejam que estudaram e aprenderam junto
comigo, ou que simplesmente estavam lá quando eu queria me
divertir. Em especial agradeço o Daniel Valesin e o Estéfano Alves de
Souza, que revisaram esse texto e me sugeriram diversas melhorias.

Por mim gostaria de agradecer a banca examinadoras, pelas correções e
melhorias sugeridas para o presente texto.

% ---------------------------------------------------------------------------- %
% Resumo
\chapter*{Resumo}

Processos K começaram a ser estudados nos anos 50 como uma fonte de
contraexemplos e de comportamento patológico. Recentemente
descobriu-se que eles são um limite de escalas para modelos de
armadilha, fato que voltou a trazer certa atenção para eles.

Nesse trabalho vamos adotar uma abordagem construtiva, usando-a para
mostrar a propriedade forte de Markov e calcular as taxas de transição
e o gerador infinitesimal.

\noindent \textbf{Palavras-chave:} processos K, estados instantâneos,
gerador infinitesimal.

% ---------------------------------------------------------------------------- %
% Abstract
\chapter*{Abstract}

K Processes were studied in the 50's as a source of counter examples
and of pathological behaviour. It is now know that they are a scaling
limit for Trap Models, which led attention back to them.

In this work, we will adopt a constructive approach, using it to show
the strong Markov propriety, calculate the transition rates and
the infinitesimal generator.

\noindent \textbf{Keywords:} K processes, instantaneous states,
infinitesimal generator.


%%% Local Variables: 
%%% mode: latex
%%% TeX-master: "tese"
%%% End: 
