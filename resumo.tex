% ---------------------------------------------------------------------------- %
% Agradecimentos
% \chapter*{Agradecimentos}
% Agradecimentos ...


% ---------------------------------------------------------------------------- %
% Resumo
\chapter*{Resumo}

Processos K começaram a ser estudados nos anos 50 como fontes de
contraexemplos e comportamento patológico. Recentemente
descobriu-se que eles são um limite de escalas para modelos de
armadilha, fato que voltou a trazer certa atenção para eles.

Nesse trabalho vamos adotar uma abordagem construtiva, usando-a para
mostrar a propriedade forte de Markov, calcular as taxas de transição
e o gerador infinitesimal.

\noindent \textbf{Palavras-chave:} processos K, estados instantâneos,
gerador infinitesimal.

% ---------------------------------------------------------------------------- %
% Abstract
\chapter*{Abstract}

K Processes were studied in the 50's as a source of counter examples
and pathological behaviour. It is now know that they are a scaling
limit for Trap-Models, which led attention back to them.

In this work, we will adopt a constructive approach, using it to show
the strong Markov propriety, calculate the transition rates and
the infinitesimal generator.

\noindent \textbf{Keywords:} k processes, instantaneous states,
infinitesimal generator.


%%% Local Variables: 
%%% mode: latex
%%% TeX-master: "tese"
%%% End: 
