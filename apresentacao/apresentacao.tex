\documentclass[xcolor=pdftex,dvipsnames]{beamer}

\usepackage[brazil]{babel}
\usepackage[T1]{fontenc}
\usepackage[utf8]{inputenc}
%\usecolortheme[named=Brown]{structure}
\usetheme{Warsaw}
\usecolortheme{crane}

\newcommand{\AAA}{{\mathcal{A}}}
\newcommand{\DDD}{{\mathfrak{D}}}
\newcommand{\FFF}{{\mathfrak{F}}}
\newcommand{\GGG}{{\mathfrak{G}}}

\newcommand{\CC}{{\mathcal{C}}}
\newcommand{\RR}{{\mathcal{R}}}
\newcommand{\II}{{\mathcal{I}}}
\newcommand{\RRb}{\widebar{\mathcal{R}}}
\newcommand{\FF}{{\mathcal{F}}}
\newcommand{\PP}{{\mathcal{P}}}
\newcommand{\GG}{{\mathcal{G}}}


\newcommand{\N}{{\mathbb{N}}}
\newcommand{\Nb}{{\widebar{\N}}}
\newcommand{\Nz}{{\mathbb{N^*}}}
\newcommand{\Nzb}{{\mathbb{\widebar{N}^*}}}
\newcommand{\Z}{{\mathbb{Z}}}
\newcommand{\R}{{\mathbb{R}}}
\newcommand{\E}{{\mathbb{E}}}


\newcommand{\qc}{{\emph{q.c.}} }
\newcommand{\ind}{{\mathbb{I}}}
\newcommand{\diam}{{\mathrm{diam}}}

\newtheorem{teorema}{Teorema}
\newtheorem{lema}[teorema]{Lema}
\newtheorem{proposicao}[teorema]{Proposição}
\newtheorem{corolario}[teorema]{Corolário}
\newtheorem{definicao}{Definição}


\title{Um estudo sobre o Processo K não homogêneo}
\author{Gabriel R. C. Peixoto \and Luiz Renato G. Fontes}
\date{22 de fevereiro de 2011}
%\institu[IME-USP]{Instituto de Matemática e Estatística}

\setbeamercovered{invisible}

\begin{document}

\begin{frame}[plain]
  \titlepage
  
  \begin{center}
    Durante o desenvolvimento deste trabalho \\
    o autor recebeu auxílio
    financeiro do CNPq.
  \end{center}
\end{frame}
% colocar apoio financeiro CNPQ

\begin{frame}{A sample slide}



A displayed formula:

\[
  \int_{-\infty}^\infty e^{-x^2} \, dx = \sqrt{\pi}
\]
\pause

An itemized list:

\begin{itemize}
  \item itemized item 1 
  \item itemized item 2 
  \item itemized item 3
\end{itemize}

\begin{teorema}
  In a right triangle, the square of hypotenuse equals the sum of
  squares of two other sides.
\end{teorema}

\end{frame}

\begin{frame}[t]{Testando}
hfdfdhfd
\begin{proof}
é trivial
\end{proof}
\end{frame}

\end{document}

%%% Local Variables: 
%%% mode: latex
%%% TeX-master: t
%%% End: 
