\documentclass[xcolor=pdftex,dvipsnames]{beamer}

\usepackage[brazil]{babel}
\usepackage[T1]{fontenc}
\usepackage[utf8]{inputenc}

\usepackage{amsmath}
\usepackage{amssymb}
\usepackage{amsthm}
\usepackage{mathabx}
\usepackage{mathrsfs}
\usepackage{yfonts}

\usetheme{Warsaw}
\usecolortheme[named=Brown]{structure}
%\usecolortheme{crane}
\setbeamertemplate{navigation symbols}{} 

\newcommand{\AAA}{{\mathcal{A}}}
\newcommand{\DDD}{{\mathfrak{D}}}
\newcommand{\FFF}{{\mathfrak{F}}}
\newcommand{\GGG}{{\mathfrak{G}}}

\newcommand{\CC}{{\mathcal{C}}}
\newcommand{\RR}{{\mathcal{R}}}
\newcommand{\II}{{\mathcal{I}}}
\newcommand{\RRb}{\widebar{\mathcal{R}}}
\newcommand{\FF}{{\mathcal{F}}}
\newcommand{\PP}{{\mathcal{P}}}
\newcommand{\GG}{{\mathcal{G}}}


\newcommand{\N}{{\mathbb{N}}}
\newcommand{\Nb}{{\widebar{\N}}}
\newcommand{\Nz}{{\mathbb{N^*}}}
\newcommand{\Nzb}{{\mathbb{\widebar{N}^*}}}
\newcommand{\Z}{{\mathbb{Z}}}
\newcommand{\R}{{\mathbb{R}}}
\newcommand{\E}{{\mathbb{E}}}


\newcommand{\qc}{{\emph{q.c.}} }
\newcommand{\ind}{{\mathbb{I}}}
\newcommand{\diam}{{\mathrm{diam}}}

\newtheorem{teorema}{Teorema}
\newtheorem{lema}[teorema]{Lema}
\newtheorem{proposicao}[teorema]{Proposição}
\newtheorem{corolario}[teorema]{Corolário}
\newtheorem{definicao}{Definição}


\title{Um estudo sobre o Processo K não homogêneo}
\author[Peixoto, G. R. C. \and Fontes L. R. G.]
{Gabriel R. C. Peixoto \and Luiz Renato G. Fontes}
\institute[IME-USP]{Instituto de Matemática e Estatística da
  Universidade de São Paulo}
\date{22 de fevereiro de 2011}


\setbeamercovered{dynamic}

\begin{document}

% Logo do IME, da USP e do CNPQ?
\begin{frame}[plain]
  \titlepage
  
  \begin{center}
    Este trabalho teve apoio do CNPq.
  \end{center}
\end{frame}

\section{Construção}


\begin{frame}{Espaço de estados}

O Processo K é um processo estocástico em tempo contínuo construído
sobre o espaço:

\begin{displaymath}
  \Nzb = \left\{
    1, 2, 3, \ldots
  \right\} \cup \{ \infty \}
\end{displaymath}

O infinito será um estado \emph{instantâneo}.

\end{frame}

\begin{frame}{Parâmetros}

  Associa-se duas constantes positivas à cada estado $x$ do processo:

  \begin{itemize}
  \item $\lambda_x$: controla a taxa em que tendemos a entrar em no
    estado $x$;
    
  \item $\gamma_x$: tempo médio em que permanecemos no estado $x$ em
    cada visita.
  \end{itemize}

  Considere ainda uma constante $c$ não negativa, que irá controlar 
  o tempo em que o processo passa no infinito.

\end{frame}

\begin{frame}{Restrições}

  \begin{columns}
    \begin{column}{0.3\textwidth}
      \begin{enumerate}
      \item $ \displaystyle \sum_{x\in \Nz} \lambda_x \gamma_x <
        \infty $    
        \bigskip
      \item $ \displaystyle \sum_{x\in \Nz} \lambda_x = \infty $
      \end{enumerate}

    \end{column}
    
    \begin{column}{0.7\textwidth}
  
      Nescessária para a existência do processo.
      
      Conveniência. No caso contrário o processo é simples.
      
    \end{column}
  \end{columns}
  
  
\end{frame}

\begin{frame}{Espaço de Probabilidade}

  O Processo K é construido num Espaço de Probabilidades que admita as
  seguintes famílias de variáveis aleatórias independentes:

  \begin{itemize}
  \item $\{ N_x: x \in \Nz\}$
  \end{itemize}
  
\end{frame}

\begin{frame}{Função $\Gamma$}
  \begin{displaymath}
    \Gamma(t) = \gamma_y T_0 
  \end{displaymath}
\end{frame}



\end{document}

%%% Local Variables: 
%%% mode: latex
%%% TeX-master: t
%%% End: 
