%% ------------------------------------------------------------------------- %%
\chapter{Introdução}
\label{cap:introducao}

Estudaremos uma família de processos markovianos em tempo contínuo sobre
um espaço de estados enumerável. A primeira menção sobre esse tipo de
processos foi dada por \cite{kolmogorov:51}. Seu objetivo nesse artigo
era dar um exemplo de um processo que tivesse um estado com taxa de
saída infinita.

\cite{kendall:56} estudaram mais a fundo o exemplo de Kolmogorov,
mostrando que as equações ``forward'' e ``backward'' podem não ser
suficientes para estudar esse processo.

Anos mais tarde, \cite{reuter:69} mostrou que apenas a matriz de taxas
desse processo é suficiente para caracterizar as probabilidades de
transição do processo.

Os autores de todos esses artigos abordaram o processo de uma maneira
bastante analítica. Em particular nenhum deles deu uma contrução
explícita para o processo estudado, se apoiando em teoremas de
existência.

Recentemente, \cite{fontes:08} revisitaram esse tipo de processo,
com a motivação de que eles aparecem como limite de escala para
\emph{Trap Models}. A maneira como abordamos o processo foi muito
inspirada nesse trabalho, em especial a construção apresentada aqui é
uma adaptação da construção deles.

Nós estudamos uma família de processos, que iremos chamar de Processos
K não homogêneos, que é uma extenção natural dos processos estudados
neste último artigo. A extenção vem do fato de admitirmos estados com
pesos não homogêneos. Uma definição precisa do que são estes pesos
será dada no Capítulo \ref{cap:construcao}.

A adição de pesos resultou em um comportamento inesperado. Existem
processos de K não homogêneos que não têm a propriedade de Feller.
Iremos explicitar quando isso acontece na Seção \ref{sec:prop-feller}.
Apesar disso, nosso processo possui propriedades que permitem que
trabalhemos com ele como se fosse um processo de Feller.

%% ------------------------------------------------------------------------- %%
\section{Contribuições}
\label{sec:contribucoes}

As principais contribuições deste trabalho são as seguintes:

\begin{itemize}

\item Não temos conhecimento de outro texto que trabalhe com processos
  K não homogêneos de maneira construtiva.

\item Calculamos explicitamente a matriz de taxas do processo K. Assim
  fornecemos uma maneira de ligar nossa construção com os textos dos
  anos 50 e 60 citados anteriormente que não use a teoria das formas
  de Dirichlet.

\end{itemize}

%% ------------------------------------------------------------------------- %%
\section{Organização do Trabalho}
\label{sec:organizacao_trabalho}

No Capítulo \ref{cap:construcao} vamos mostrar uma construção do
processo K não homogêneo e colocar algumas definições básicas.

O Capítulo \ref{cap:propriedades} vai nos fornecer ferramentas para
trabalhar. Nele vamos dar uma maneira de aproximar processos K por
processos Markovianos de Saltos e provar a propriedade (forte) de
Markov.

Por fim no Capítulo \ref{cap:taxas} vamos nos focar nas probabilidades
de transição. Vamos mostrar que elas são funções contínuas no tempo e
calcular suas derivadas na origem. Por fim vamos calcular a medida
invariante do processo e dar um resultado relativo à dimensão de
Hausdorf do tempo em que o processo passa no infinito.


%%% Local Variables: 
%%% TeX-master: "tese"
%%% End: 
