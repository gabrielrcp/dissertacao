%% ------------------------------------------------------------------------- %%
\chapter{Introdução}
\label{cap:introducao}

%% ------------------------------------------------------------------------- %%
\section{Considerações Preliminares}
\label{sec:consideracoes_preliminares}

Estudaremos uma família de processos markovianos em tempo contínuo
sobre um espaço de estados enumerável. A primeira menção desse tipo de
processos foi dada por \cite{kolmogorov:51}. Seu objetivo era dar um
exemplo de um processo que tivesse um estado com taxa de saída
infinita.

\cite{kendall:56} estudaram mais a fundo o exemplo de Kolmogorov,
encontrando equações \emph{forward} e \emph{backward} para as
probabilidades de transição.

Anos mais tarde, \cite{reuter:69} mostrou que apenas a matriz de taxas
desse processo é suficiente para caracterizar as probabilidades de
transição do exemplo de Kolmogorov. Ele também introduziu a não
homogeneidade.

Os autores desses artigos abordaram o processo de uma maneira
analítica. Em particular nenhum deles deu uma construção explícita
para o processo estudado, se apoiando em teoremas de existência.

Recentemente, \cite{fontes:08} revisitaram esse tipo de processo com a
motivação de que eles aparecem como limite de escala para Modelos de
Armadilha. A maneira como abordamos o processo foi muito inspirada
nesse trabalho e, em especial, a construção apresentada é uma
adaptação da construção deles.

%% ------------------------------------------------------------------------- %%
%\section{Objetivos}
%\label{sec:objetivo}

Nós estudamos uma família de processos que iremos chamar de Processos
K não homogêneos, que é uma extensão natural dos processos estudados
neste último artigo. A extensão vem do fato de darmos um peso para
cada estado. Uma definição precisa do que são estes pesos será
apresentada no Capítulo \ref{cap:construcao}.

A adição de pesos resultou em um comportamento inesperado. Existem
Processos K não homogêneos que não apresentam a propriedade de Feller.
Iremos explicitar sobre quais condições isso acontece na Seção
\ref{sec:prop-feller}.  Apesar disso, essa família possui propriedades
que permitem que trabalhemos quase como se estivéssemos com processos
de Feller.

%% ------------------------------------------------------------------------- %%
\section{Contribuições}
\label{sec:contribucoes}

As principais contribuições deste trabalho são as seguintes:

\begin{itemize}

\item Não temos conhecimento de outro texto que trabalhe com Processos
  K não homogêneos de maneira construtiva.

\item Calculamos explicitamente a matriz de taxas do Processo K. Assim
  fornecemos uma maneira de ligar nossa construção, quando $c > 0$, com
  os textos dos anos 50 e 60 citados anteriormente que não use a
  teoria das formas de Dirichlet.

\item Encontramos o gerador infinitesimal no caso homogêneo para $c =
  0$.

\end{itemize}

%% ------------------------------------------------------------------------- %%
\section{Organização do Trabalho}
\label{sec:organizacao_trabalho}

No Capítulo \ref{cap:construcao} vamos construir o Processo K não
homogêneo e apresentar algumas definições básicas.

O Capítulo \ref{cap:propriedades} vai nos fornecer ferramentas para
trabalhar. Nele vamos apresentar uma maneira de aproximar processos K por
processos Markovianos de Saltos e provar a propriedade forte de
Markov.

No Capítulo \ref{cap:taxas} vamos focar nas probabilidades de
transição. Mostraremos que elas são funções contínuas no tempo e
calcularemos suas derivadas na origem. Isso nos permitirá encontrar a
medida invariante do processo.  Aprofundando um pouco mais no cálculo
de taxas de transição, exibiremos o gerador infinitesimal do processo
no caso homogêneo.

Por fim mostraremos um resultado relativo à dimensão de Hausdorff do
tempo em que o processo passa no infinito.


%%% Local Variables: 
%%% TeX-master: "tese"
%%% End: 
