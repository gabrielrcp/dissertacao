%% ------------------------------------------------------------------------- %%
\chapter{Taxas de Transição}
\label{cap:taxas}

Vamos começar a escrever e depois penso no que escrever aqui.

%% ------------------------------------------------------------------------- %%

\section{Continuidade}
\label{sec:continuidade}

Para deixar a nossa notação menos carregada, vamos introduzir a
seguinte definição, para $x, y \in \Nzb$, $t \geq 0$:
\begin{equation}
  p_{xy} (t) = P(X^x(t) = y).
\end{equation}
Denotaremos ainda por $P(t)$ a ``matriz'' cujas entradas sejam $p_{xy}(t)$.

Nessa seção vamos estudar a continuidade ou não continuidade dessa
função na origem.

\begin{proposicao}
  Para $x \in \Nz$,
  \begin{equation}
    \lim_{t \searrow 0}p_{xx}(t) = 1.   
  \end{equation}
\end{proposicao}
\begin{proof}
  Como $x \in \Nz$, temos que:
  \begin{displaymath}
    p_{xx}(t) \geq P( \gamma_x T^x_0 > t) = e^{-\frac{t}{\gamma_x}}
    \to 1.
  \end{displaymath}
\end{proof}

\begin{lema}
  \label{lema:deriv_gamma}
  Vale que $ \frac{\Gamma^{\infty}(t)}{t} \xrightarrow{t \searrow 0}
  c$ em probabilidade.
\end{lema}
\begin{proof}
  Fixando um $t > 0$, teremos que:
  \begin{equation}
    \label{eq:gamma_tt}
    \frac{\Gamma^{\infty}(t)}{t} = 
    c + \frac{1}{t} \sum_{x \in \Nz} \sum_{i = 1}^{N^x(t)}
    \gamma_x T^x_i
  \end{equation}

  Agora podemos calcular a transformada de Laplace da segunda parte
  dessa expressão. Fixado um $u > 0$, teremos que:
  \begin{equation}
    \label{eq:lap_vaizero}
    \E \left[ \exp\left\{
      -u \frac{1}{t} \sum_{x \in \Nz} \sum_{i=1}^{N^x(t)} \gamma_x T^x_i
    \right\} \right] =
    \exp \left\{
      - u \sum_{x \in \Nz}  \frac{\lambda_x \gamma_x}{1 +
        \frac{u \gamma_x}{t}}
    \right\} %\xrightarrow{t \searrow 0} 1.
  \end{equation}

  Essa função é monótona em $t$ e cada termo da soma
  vai a zero quando $t \searrow 0$.  Assim o teorema da convergência
  monótona nos diz que \ref{eq:lap_vaizero} converge à $1$ quanto $t
  \searrow 0$, para qualquer $u \geq 0$ fixado.

  Como a função constante igual a $1$ é a transformada de Laplace de
  uma variável aleatória que vale $0$ sempre, temos que a expressão
  \ref{eq:gamma_tt} converge em probabilidade para $c$ quando $t
  \searrow 0$.
\end{proof}

\begin{lema}
  \label{lema:deriv_inv_gamma}
  Seja $\Gamma^{*}$  a função inversa de $\Gamma^\infty$.
  Vale que $ \frac{\Gamma^{*}(t)}{t} \xrightarrow{t \searrow 0}
  \frac{1}{c}$ em probabilidade.
\end{lema}
\begin{proof}

  Fixe $t > 0$ e $\epsilon > 0$. Com probabilidade $1$, temos que $t$
  é um ponto de continuidade de $\Gamma^\infty$. Assim existe $s > 0$
  tal que $\Gamma$

\end{proof}

\begin{proposicao}
  \label{prop:continuidade}
  Caso $c > 0$, vale que:
  \begin{displaymath}
    \lim_{t \searrow 0}p_{\infty \infty}(t) = 1.    
  \end{displaymath}
\end{proposicao}

\begin{proof}
  Considere a função:
  \begin{displaymath}
    \theta(t) = \int_0^t \ind \{ X^\infty (s) = \infty \} ds.
  \end{displaymath}.
  
  Pela construção do processo, temos que $\theta(t) \geq c
  \Gamma^*(t)$, onde $\Gamma^*(t)$ é a inversa de $\Gamma^\infty(t)$.

  O Lema \ref{lema:deriv_gamma} diz que $\frac{\Gamma^\infty(t)}{t}$
  converge à $1$ em probabilidade. Assim podemos reconstruir
  $\frac{\Gamma^\infty(t)}{t}$ em um novo espaço de probabilidades de
  forma que elas convirjam quase certamente.

  Agora vamos mostrar que essa expressão converge em probabilidade
  para $1$. Para isso vamos calcular a transformada de Laplace de
  $\frac{1}{ct} \sum_{x \in \Nz} \sum_{i = 1}^{N^x(t)} \gamma_x
  T^x_i$. Fixado um $u \geq 0$, podemos calcular:



  % Assim podemos reconstruir essas variáveis em um novo espaço de
  % probabilidade de forma que elas convirjam quase certamente. Como
  % $\frac{\theta(\Gamma^\infty(t))}{\Gamma^\infty(t)} \leq 1$, usando o
  % teorema da convergência dominada teremos que $$

  % Voltar para ca depois!!!!!!


  Agora, usando o Teorema de Fubbini, teremos que:
  \begin{align*}
    1 &= \lim_{t \searrow 0} \E \left[ \frac{\theta(t)}{t} \right] \\ 
    &= \lim_{t \searrow 0} \E\left[
      \frac{1}{t} \int_0^t \ind\{ X^\infty(s) = \infty \} ds
    \right] \\
    &= \lim_{t \searrow 0} 
      \frac{1}{t} \int_0^t \E\left[ \ind\{ X^\infty(s) = \infty \}\right] ds
    \\
    &= \lim_{t \searrow 0} \frac{1}{t} \int_0^t p_{\infty \infty} (s) ds
  \end{align*}

  Portanto, usando o teorema fundamental do calculo, teremos que
  $p_{\infty \infty} (t) \xrightarrow{t \searrow 0} 0$


\end{proof}

\begin{proposicao}
  \label{prop:naocontinuidade}
  Caso $c = 0$, para todo $x \in \Nzb$ e $t > 0$, temos que $p_{x
    \infty} (t) = 0$.
\end{proposicao}
\begin{proof}
\end{proof}


%% ------------------------------------------------------------------------- %%

\section{Matriz Q}
\label{sec:matrizq}

Nessa seção vamos calcular a matriz de taxas de transição do nosso
processo. Ela pode ser definida pelo limite:
\begin{displaymath}
  Q = \lim_{t \searrow 0} \frac{P(t) - I}{t} 
\end{displaymath}.

Vamos mostrar que, para o caso $c > 0$, essa matriz vale:
\begin{displaymath}
  Q = \left(
    \begin{array}{ccccc}
      -\frac{1}{\gamma_1} & 0 & 0 & \cdots & \frac{1}{\gamma_1}\\
      0 & -\frac{1}{\gamma_2} & 0 & \cdots & \frac{1}{\gamma_2}\\
      0 & 0 & -\frac{1}{\gamma_3} & \cdots & \frac{1}{\gamma_3}\\
      \vdots & \vdots & \vdots & \vdots & \ddots \\
      \frac{\lambda_1}{c} & \frac{\lambda_2}{c} &
      \frac{\lambda_3}{c} & \cdots & -\infty\\
    \end{array}
  \right).
\end{displaymath}

Equanto que no caso $c=0$, teremos:
\begin{displaymath}
  Q = \left(
    \begin{array}{ccccc}
      -\frac{1}{\gamma_1} & 0 & 0 & \cdots & 0\\
      0 & -\frac{1}{\gamma_2} & 0 & \cdots & 0\\
      0 & 0 & -\frac{1}{\gamma_3} & \cdots & 0\\
      \vdots & \vdots & \vdots & \vdots & \ddots \\
      \infty & \infty & \infty & \cdots & -\infty\\
    \end{array}
  \right).
\end{displaymath}

\begin{proposicao}
  Sejam $x, y \in \Nz$, $x \neq y$, vale que:
  \begin{displaymath}
    \lim_{t \searrow 0} \frac{p_{xy}(t)}{t} = 0.
  \end{displaymath}
\end{proposicao}
\begin{proof}
  \begin{align*}
    \frac{p_{x y} (t)}{t}
    &= \frac{1}{t}\int_{0}^{t} P( X^x(t) = y |
    \gamma_x T_0^x = s) \frac{1}{\gamma_x} e^{-\frac{s}{\gamma_x}} ds\\
    &= \frac{1}{t} \int_{0}^{t} P( X^\infty(t-s) = y ) \frac{1}{\gamma_x}
    e^{-\frac{s}{\gamma_x}} ds \\
    &\leq \frac{1}{t \gamma_x} \int_{0}^{t} p_{\infty y}(t-s) ds \\
    &= \frac{1}{t \gamma_x} \int_{0}^{t} p_{\infty y}(t-s) ds \\
    &= \frac{1}{t \gamma_x} \int_{0}^{t} p_{\infty y}(s) ds
    \xrightarrow{t \searrow 0} \lim_{t \searrow 0} \frac{p_{\infty y}
      (t)}{\gamma_x} = 0.
  \end{align*}

\end{proof}

\begin{proposicao}
  Para $x \in \Nz$, vale que:
  \begin{displaymath}
    \lim_{t \searrow 0} \frac{p_{xx}(t) - 1}{t} = -\frac{1}{\gamma_x}
  \end{displaymath}
\end{proposicao}
\begin{proof}
  \begin{align*}
    p_{xx} (t)
    &= P( \gamma_x T_0^x > t) + 
    \int_{0}^{t} P( X^x(t) = y |
    \gamma_x T_0^x = s) \frac{1}{\gamma_x} e^{-\frac{s}{\gamma_x}} ds\\
    &= e^{-\frac{t}{\gamma_x}} + 
    \int_{0}^{t} P( X^\infty(t-s) = y) \frac{1}{\gamma_x} e^{-\frac{s}{\gamma_x}} ds\\
  \end{align*}
  Fazendo contas análogas às da proposição anterior, podemos
  mostrar que o segundo termo dessa soma, dividido por $t$ vai para
  zero quando $t \searrow 0$. Tratando o primeiro termo agora, temos
  que:
  \begin{displaymath}
    \frac{e^{-\frac{t}{\gamma_x}} - 1}{t} \xrightarrow{t \searrow 0}
    -\frac{1}{\gamma_x}.
  \end{displaymath}

  Dessa forma:
  \begin{displaymath}
     \lim_{t \searrow 0} \frac{p_{xx} (t) - 1}{t} = -\frac{1}{\gamma_x}
  \end{displaymath}
  
\end{proof}

\begin{proposicao}
  Para $x \in \Nz$, vale que:
  \begin{displaymath}
    \lim_{t \searrow 0} \frac{p_{\infty x}(t)}{t} = \begin{cases}
      \frac{\lambda_x}{c} & \textrm{ se } c > 0 \\
      \infty & \textrm{ se } c = 0 \\
    \end{cases}
  \end{displaymath}
\end{proposicao}
\begin{proof}

\end{proof}

\begin{proposicao}
  Vale que:
  \begin{displaymath}
    \lim_{t \searrow 0} \frac{p_{\infty \infty}(t) - 1}{t} = -\infty
  \end{displaymath}
\end{proposicao}
\begin{proof}
  
\end{proof}

\begin{proposicao}
  Para $x \in \Nz$, vale que:
  \begin{displaymath}
    \lim_{t \searrow 0} \frac{p_{x \infty}(t)}{t} = \begin{cases}
      \frac{1}{\gamma_x} & \textrm{ se } c > 0 \\
      0 & \textrm{ se } c = 0 \\
    \end{cases}
  \end{displaymath}
\end{proposicao}
\begin{proof}
  
\end{proof}

%% ------------------------------------------------------------------------- %%

\section{Medida invariante}
\label{sec:invariante}

O objetivo dessa seção é mostrar que a seguinte distrubuição de
probabilidade é uma medida invariante para o processo K.

\begin{equation}
  \label{eq:invariante}
  \pi(x) = \begin{cases}
    \frac{\lambda_x \gamma_x}{c + \sum_{y \in \Nz} \lambda_y \gamma_y}
    & \textrm{ se } x \in \Nz \\
    \frac{c}{c + \sum_{y \in \Nz} \lambda_y \gamma_y}
    & \textrm{ se } x = \infty \\
  \end{cases}
\end{equation}

\begin{proposicao}
  O processo K tem uma única medida invariante.
\end{proposicao}
\begin{proof}
  
\end{proof}


\begin{proposicao}
  A probabilidade $\pi$ definida em \ref{eq:invariante} é a medida
  invariante do processo K, caso $c > 0$.
\end{proposicao}
\begin{proof}
  
\end{proof}

\begin{proposicao}
  A probabilidade $\pi$ definida em \ref{eq:invariante} é a medida
  invariante do processo K, caso $c = 0$.
\end{proposicao}
\begin{proof}
  
\end{proof}


%%% Local Variables: 
%%% TeX-master: "tese"
%%% End: 
