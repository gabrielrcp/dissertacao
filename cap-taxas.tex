%% ------------------------------------------------------------------------- %%
\chapter{Taxas de Transição}
\label{cap:taxas}

Vamos começar a escrever e depois penso no que escrever aqui.

%% ------------------------------------------------------------------------- %%

\section{Continuidade}
\label{sec:continuidade}

Para deixar a nossa notação menos carregada, vamos introduzir a
seguinte definição, para $x, y \in \Nzb$, $t \geq 0$:
\begin{equation}
  p_{xy} (t) = P(X^x(t) = y).
\end{equation}
Denotaremos ainda por $P(t)$ a ``matriz'' cujas entradas sejam $p_{xy}(t)$.

Nessa seção vamos estudar a continuidade ou não continuidade dessa
função na origem.

\begin{proposicao}
  Para $x \in \Nz$,
  \begin{equation}
    \lim_{t \searrow 0}p_{xx}(t) = 1.   
  \end{equation}
\end{proposicao}
\begin{proof}
  Como $x \in \Nz$, temos que:
  \begin{displaymath}
    p_{xx}(t) \geq P( \gamma_x T^x_0 > t) = e^{-\frac{t}{\gamma_x}}
    \to 1.
  \end{displaymath}
\end{proof}

\begin{proposicao}
  \label{prop:naocontinuidade}
  Caso $c = 0$, para todo $x \in \Nzb$ e $t > 0$, temos que $p_{x
    \infty} (t) = 0$.
\end{proposicao}
\begin{proof}
\end{proof}


\begin{proposicao}
  \label{prop:continuidade}
  Caso $c > 0$, vale que:
  \begin{displaymath}
    \lim_{t \searrow 0}p_{\infty \infty}(t) = 1.    
  \end{displaymath}
\end{proposicao}

%% ------------------------------------------------------------------------- %%

\section{Matriz Q}
\label{sec:matrizq}

Nessa seção vamos calcular a matriz de taxas de transição do nosso
processo. Ela pode ser definida pelo limite:
\begin{displaymath}
  Q = \lim_{t \searrow 0} \frac{P(t) - I}{t} 
\end{displaymath}.

Vamos mostrar que, para o caso $c > 0$, essa matriz vale:
\begin{displaymath}
  Q = \left(
    \begin{array}{ccccc}
      \frac{1}{\gamma_1} & -\frac{1}{\gamma_1} & 0 & 0 & \cdots \\
      \frac{1}{\gamma_2} & 0 & -\frac{1}{\gamma_2} & 0 & \cdots \\
      \frac{1}{\gamma_3} & 0 & 0 & -\frac{1}{\gamma_3} & \cdots \\
      \vdots & \vdots & \vdots & \vdots & \ddots \\
      -\infty & \frac{\lambda_1}{c} & \frac{\lambda_2}{c} &
      \frac{\lambda_3}{c} & \cdots \\
    \end{array}
  \right).
\end{displaymath}

Equanto que no caso $c=0$, teremos:
\begin{displaymath}
  Q = \left(
    \begin{array}{ccccc}
      0 & -\frac{1}{\gamma_1} & 0 & 0 & \cdots \\
      0 & 0 & -\frac{1}{\gamma_2} & 0 & \cdots \\
      0 & 0 & 0 & -\frac{1}{\gamma_3} & \cdots \\
      \vdots & \vdots & \vdots & \vdots & \ddots \\
      -\infty & \infty & \infty & \infty & \cdots \\
    \end{array}
  \right).
\end{displaymath}

\begin{proposicao}
  Sejam $x, y \in \Nz$, $x \neq y$, vale que:
  \begin{displaymath}
    \lim_{t \searrow 0} \frac{p_{xy}(t)}{t} = 0.
  \end{displaymath}
\end{proposicao}
\begin{proof}
  
\end{proof}

\begin{proposicao}
  Para $x \in \Nz$, vale que:
  \begin{displaymath}
    \lim_{t \searrow 0} \frac{p_{xx}(t) - 1}{t} = -\frac{1}{\gamma_x}
  \end{displaymath}
\end{proposicao}
\begin{proof}
  
\end{proof}

\begin{proposicao}
  Para $x \in \Nz$, vale que:
  \begin{displaymath}
    \lim_{t \searrow 0} \frac{p_{\infty x}(t)}{t} = \begin{cases}
      \frac{\lambda_x}{c} & \textrm{ se } c > 0 \\
      \infty & \textrm{ se } c = 0 \\
    \end{cases}
  \end{displaymath}
\end{proposicao}
\begin{proof}
  
\end{proof}

\begin{proposicao}
  Vale que:
  \begin{displaymath}
    \lim_{t \searrow 0} \frac{p_{\infty \infty}(t) - 1}{t} = -\infty
  \end{displaymath}
\end{proposicao}
\begin{proof}
  
\end{proof}

\begin{proposicao}
  Para $x \in \Nz$, vale que:
  \begin{displaymath}
    \lim_{t \searrow 0} \frac{p_{x \infty}(t)}{t} = \begin{cases}
      \frac{1}{\gamma_x} & \textrm{ se } c > 0 \\
      0 & \textrm{ se } c = 0 \\
    \end{cases}
  \end{displaymath}
\end{proposicao}
\begin{proof}
  
\end{proof}

%% ------------------------------------------------------------------------- %%

\section{Medida invariante}
\label{sec:invariante}

O objetivo dessa seção é mostrar que a seguinte distrubuição de
probabilidade é uma medida invariante para o processo K.

\begin{equation}
  \label{eq:invariante}
  \pi(x) = \begin{cases}
    \frac{\lambda_x \gamma_x}{c + \sum_{y \in \Nz} \lambda_y \gamma_y}
    & \textrm{ se } x \in \Nz \\
    \frac{c}{c + \sum_{y \in \Nz} \lambda_y \gamma_y}
    & \textrm{ se } x = \infty \\
  \end{cases}
\end{equation}

\begin{proposicao}
  O processo K tem uma única medida invariante.
\end{proposicao}
\begin{proof}
  
\end{proof}


\begin{proposicao}
  A probabilidade $\pi$ definida em \ref{eq:invariante} é a medida
  invariante do processo K, caso $c > 0$.
\end{proposicao}
\begin{proof}
  
\end{proof}

\begin{proposicao}
  A probabilidade $\pi$ definida em \ref{eq:invariante} é a medida
  invariante do processo K, caso $c = 0$.
\end{proposicao}
\begin{proof}
  
\end{proof}


%%% Local Variables: 
%%% TeX-master: "tese"
%%% End: 
