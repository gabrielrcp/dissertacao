%% ------------------------------------------------------------------------- %%
\chapter{Taxas de Transição}
\label{cap:taxas}

Vamos começar a escrever e depois penso no que escrever aqui.

%% ------------------------------------------------------------------------- %%

\section{Continuidade}
\label{sec:continuidade}

Para deixar a nossa notação menos carregada, vamos introduzir a
seguinte definição, para $x, y \in \Nzb$, $t \geq 0$:
\begin{equation}
  p_{xy} (t) = P(X^x(t) = y).
\end{equation}
Denotaremos ainda por $P(t)$ a ``matriz'' cujas entradas sejam $p_{xy}(t)$.

Nessa seção vamos estudar a continuidade ou não continuidade dessa
função na origem.

\begin{proposicao}
  Para $x \in \Nz$,
  \begin{equation}
    \lim_{t \searrow 0}p_{xx}(t) = 1.   
  \end{equation}
\end{proposicao}
\begin{proof}
  Como $x \in \Nz$, temos que:
  \begin{displaymath}
    p_{xx}(t) \geq P( \gamma_x T^x_0 > t) = e^{-\frac{t}{\gamma_x}}
    \to 1.
  \end{displaymath}
\end{proof}

\begin{lema}
  \label{lema:deriv_gamma}
  Vale que $ \frac{\Gamma^{\infty}(t)}{t} \xrightarrow{t \searrow 0}
  c$ em probabilidade.
\end{lema}
\begin{proof}
  Fixando um $t > 0$, teremos que:
  \begin{equation}
    \label{eq:gamma_tt}
    \frac{\Gamma^{\infty}(t)}{t} = 
    c + \frac{1}{t} \sum_{x \in \Nz} \sum_{i = 1}^{N^x(t)}
    \gamma_x T^x_i
  \end{equation}

  Agora podemos calcular a transformada de Laplace da segunda parte
  dessa expressão. Fixado um $u > 0$, teremos que:
  \begin{equation}
    \label{eq:lap_vaizero}
    \E \left[ \exp\left\{
      -u \frac{1}{t} \sum_{x \in \Nz} \sum_{i=1}^{N^x(t)} \gamma_x T^x_i
    \right\} \right] =
    \exp \left\{
      - u \sum_{x \in \Nz}  \frac{\lambda_x \gamma_x}{1 +
        \frac{u \gamma_x}{t}}
    \right\} %\xrightarrow{t \searrow 0} 1.
  \end{equation}

  Essa função é monótona em $t$ e cada termo da soma
  vai a zero quando $t \searrow 0$.  Assim o teorema da convergência
  monótona nos diz que \ref{eq:lap_vaizero} converge à $1$ quanto $t
  \searrow 0$, para qualquer $u \geq 0$ fixado.

  Como a função constante igual a $1$ é a transformada de Laplace de
  uma variável aleatória que vale $0$ sempre, temos que a expressão
  \ref{eq:gamma_tt} converge em probabilidade para $c$ quando $t
  \searrow 0$.
\end{proof}

\begin{lema}
  \label{lema:deriv_inv_gamma}
  Seja $\Gamma^{*}$  a função inversa de $\Gamma^\infty$.
  Vale que $ \frac{\Gamma^{*}(t)}{t} \xrightarrow{t \searrow 0}
  \frac{1}{c}$ em probabilidade.
\end{lema}
\begin{proof}

  Fixe $t > 0$ e $\epsilon > 0$. Com probabilidade $1$, temos que $t$
  é um ponto de continuidade de $\Gamma^\infty$. Assim existe $s > 0$
  tal que $\Gamma$

\end{proof}

\begin{proposicao}
  \label{prop:continuidade}
  Caso $c > 0$, vale que:
  \begin{displaymath}
    \lim_{t \searrow 0}p_{\infty \infty}(t) = 1.    
  \end{displaymath}
\end{proposicao}

\begin{proof}
  Considere a função:
  \begin{displaymath}
    \theta(t) = \int_0^t \ind \{ X^\infty (s) = \infty \} ds.
  \end{displaymath}.
  
  Pela construção do processo, temos que $\theta(t) \geq c
  \Gamma^*(t)$, onde $\Gamma^*(t)$ é a inversa de $\Gamma^\infty(t)$.

  O Lema \ref{lema:deriv_gamma} diz que $\frac{\Gamma^\infty(t)}{t}$
  converge à $1$ em probabilidade. Assim podemos reconstruir
  $\frac{\Gamma^\infty(t)}{t}$ em um novo espaço de probabilidades de
  forma que elas convirjam quase certamente.

  Agora vamos mostrar que essa expressão converge em probabilidade
  para $1$. Para isso vamos calcular a transformada de Laplace de
  $\frac{1}{ct} \sum_{x \in \Nz} \sum_{i = 1}^{N^x(t)} \gamma_x
  T^x_i$. Fixado um $u \geq 0$, podemos calcular:



  % Assim podemos reconstruir essas variáveis em um novo espaço de
  % probabilidade de forma que elas convirjam quase certamente. Como
  % $\frac{\theta(\Gamma^\infty(t))}{\Gamma^\infty(t)} \leq 1$, usando o
  % teorema da convergência dominada teremos que $$

  % Voltar para ca depois!!!!!!


  Agora, usando o Teorema de Fubbini, teremos que:
  \begin{align*}
    1 &= \lim_{t \searrow 0} \E \left[ \frac{\theta(t)}{t} \right] \\ 
    &= \lim_{t \searrow 0} \E\left[
      \frac{1}{t} \int_0^t \ind\{ X^\infty(s) = \infty \} ds
    \right] \\
    &= \lim_{t \searrow 0} 
      \frac{1}{t} \int_0^t \E\left[ \ind\{ X^\infty(s) = \infty \}\right] ds
    \\
    &= \lim_{t \searrow 0} \frac{1}{t} \int_0^t p_{\infty \infty} (s) ds
  \end{align*}

  Portanto, usando o teorema fundamental do calculo, teremos que
  $p_{\infty \infty} (t) \xrightarrow{t \searrow 0} 0$


\end{proof}

\begin{proposicao}
  \label{prop:naocontinuidade}
  Caso $c = 0$, para todo $x \in \Nzb$ e $t > 0$, temos que $p_{x
    \infty} (t) = 0$.
\end{proposicao}
\begin{proof}
\end{proof}


%% ------------------------------------------------------------------------- %%

\section{Matriz Q}
\label{sec:matrizq}

Nessa seção vamos calcular a matriz de taxas de transição do nosso
processo. Ela pode ser definida pelo limite:
\begin{displaymath}
  Q = \lim_{t \searrow 0} \frac{P(t) - I}{t} 
\end{displaymath}.

Vamos mostrar que, para o caso $c > 0$, essa matriz vale:
\begin{displaymath}
  Q = \left(
    \begin{array}{ccccc}
      -\frac{1}{\gamma_1} & 0 & 0 & \cdots & \frac{1}{\gamma_1}\\
      0 & -\frac{1}{\gamma_2} & 0 & \cdots & \frac{1}{\gamma_2}\\
      0 & 0 & -\frac{1}{\gamma_3} & \cdots & \frac{1}{\gamma_3}\\
      \vdots & \vdots & \vdots & \vdots & \ddots \\
      \frac{\lambda_1}{c} & \frac{\lambda_2}{c} &
      \frac{\lambda_3}{c} & \cdots & -\infty\\
    \end{array}
  \right).
\end{displaymath}

Equanto que no caso $c=0$, teremos:
\begin{displaymath}
  Q = \left(
    \begin{array}{ccccc}
      -\frac{1}{\gamma_1} & 0 & 0 & \cdots & 0\\
      0 & -\frac{1}{\gamma_2} & 0 & \cdots & 0\\
      0 & 0 & -\frac{1}{\gamma_3} & \cdots & 0\\
      \vdots & \vdots & \vdots & \vdots & \ddots \\
      \infty & \infty & \infty & \cdots & -\infty\\
    \end{array}
  \right).
\end{displaymath}

\begin{proposicao}
  Sejam $x, y \in \Nz$, $x \neq y$, vale que:
  \begin{displaymath}
    \lim_{t \searrow 0} \frac{p_{xy}(t)}{t} = 0.
  \end{displaymath}
\end{proposicao}
\begin{proof}
  \begin{align*}
    \frac{p_{x y} (t)}{t}
    &= \frac{1}{t}\int_{0}^{t} P( X^x(t) = y |
    \gamma_x T_0^x = s) \frac{1}{\gamma_x} e^{-\frac{s}{\gamma_x}} ds\\
    &= \frac{1}{t} \int_{0}^{t} P( X^\infty(t-s) = y ) \frac{1}{\gamma_x}
    e^{-\frac{s}{\gamma_x}} ds \\
    &\leq \frac{1}{t \gamma_x} \int_{0}^{t} p_{\infty y}(t-s) ds \\
    &= \frac{1}{t \gamma_x} \int_{0}^{t} p_{\infty y}(t-s) ds \\
    &= \frac{1}{t \gamma_x} \int_{0}^{t} p_{\infty y}(s) ds
    \xrightarrow{t \searrow 0} \lim_{t \searrow 0} \frac{p_{\infty y}
      (t)}{\gamma_x} = 0.
  \end{align*}

\end{proof}

\begin{proposicao}
  Para $x \in \Nz$, vale que:
  \begin{displaymath}
    \lim_{t \searrow 0} \frac{p_{xx}(t) - 1}{t} = -\frac{1}{\gamma_x}
  \end{displaymath}
\end{proposicao}
\begin{proof}
  \begin{align*}
    p_{xx} (t)
    &= P( \gamma_x T_0^x > t) + 
    \int_{0}^{t} P( X^x(t) = y |
    \gamma_x T_0^x = s) \frac{1}{\gamma_x} e^{-\frac{s}{\gamma_x}} ds\\
    &= e^{-\frac{t}{\gamma_x}} + 
    \int_{0}^{t} P( X^\infty(t-s) = y) \frac{1}{\gamma_x} e^{-\frac{s}{\gamma_x}} ds\\
  \end{align*}
  Fazendo contas análogas às da proposição anterior, podemos
  mostrar que o segundo termo dessa soma, dividido por $t$ vai para
  zero quando $t \searrow 0$. Tratando o primeiro termo agora, temos
  que:
  \begin{displaymath}
    \frac{e^{-\frac{t}{\gamma_x}} - 1}{t} \xrightarrow{t \searrow 0}
    -\frac{1}{\gamma_x}.
  \end{displaymath}

  Dessa forma:
  \begin{displaymath}
     \lim_{t \searrow 0} \frac{p_{xx} (t) - 1}{t} = -\frac{1}{\gamma_x}
  \end{displaymath}
  
\end{proof}

\begin{proposicao}
  Para $x \in \Nz$, vale que:
  \begin{displaymath}
    \lim_{t \searrow 0} \frac{p_{\infty x}(t)}{t} = \begin{cases}
      \frac{\lambda_x}{c} & \textrm{ se } c > 0 \\
      \infty & \textrm{ se } c = 0 \\
    \end{cases}
  \end{displaymath}
\end{proposicao}
\begin{proof}
  \begin{align}
    \frac{p_{\infty x}}{t} &= \frac{1}{t} P \left( \bigcup_{i =
        1}^{\infty} \left\{ \Gamma^\infty (\sigma^x_i -) \leq t <
        \Gamma^\infty(\sigma^x_i) \right\} \right) \notag \\
    &= \frac{P \left( \Gamma^\infty (\sigma^x_1 -) \leq t <
      \Gamma^\infty(\sigma^x_1) \right)}{t} +
    \frac{1}{t} P \left( \bigcup_{i =
        2}^{\infty} \left\{ \Gamma^\infty (\sigma^x_i -) \leq t <
        \Gamma^\infty(\sigma^x_i) \right\} \right) \notag \notag \\
    \label{erros_taxa_inf}
    &= \frac{P \left( \Gamma^\infty (\sigma^x_1 -) \leq t \right)}{t} -
    \frac{P \left( \Gamma^\infty (\sigma^x_1) \leq t \right)}{t} +
    \frac{1}{t} P \left( \bigcup_{i =
        2}^{\infty} \left\{ \Gamma^\infty (\sigma^x_i -) \leq t <
        \Gamma^\infty(\sigma^x_i) \right\} \right)
  \end{align}

  Agora vamos mostrar que o segundo termo dessa soma vai a zero quando
  $t \searrow 0$.
  \begin{align*}
    \frac{P (\Gamma^\infty (\sigma^x_1) \leq t)}{t}
    &= \frac{1}{t} P \left(
      \gamma_x T^x_1 + 
      \sum_{y \neq x} \sum_{i = 1}^{N_y (\sigma^x_1)} \gamma_y T^y_i +
      c\sigma^x_1
      \leq t
    \right) \\
    &\leq \frac{1}{t} P \left(
      \gamma_x T^x_1 + 
      \sum_{y \neq x} \sum_{i = 1}^{N_y (\sigma^x_1)} \gamma_y T^y_i
      \leq t
    \right)\\
    &= \frac{1}{t} \int_0^t P \left(
      \sum_{y \neq x} \sum_{i = 1}^{N_y (\sigma^x_1)} \gamma_y T^y_i
      \leq t - s
      \middle\vert \gamma_x T^x_1 = s
    \right) \frac{1}{\gamma_x} e^{-\frac{s}{\gamma_x}} ds\\
    &\leq \frac{1}{\gamma_x t} \int_0^t P \left(
      \sum_{y \neq x} \sum_{i = 1}^{N_y (\sigma^x_1)} \gamma_y T^y_i
      \leq t - s
    \right) ds\\
    &\leq \frac{1}{\gamma_x t} \int_0^t P \left(
      \sum_{y \neq x} \sum_{i = 1}^{N_y (\sigma^x_1)} \gamma_y T^y_i
      \leq s
    \right) ds\\
    &\xrightarrow{t\searrow0} \frac{1}{\gamma_x} P \left(
      \sum_{y \neq x} \sum_{i = 1}^{N_y (\sigma^x_1)} \gamma_y T^y_i
      = 0 \right) = 0
  \end{align*}

  Agora note que, como $\Gamma^\infty$ é não decrescente, o evento no
  terceiro termo de \eqref{erros_taxa_inf} está contido no evento do
  segundo termo. Assim como mostramos que o segundo termo vai a zero,
  teremos que o terceiro também irá.

  Agora resta calcular o limite do primeiro termo. Ao fazer isso
  estaremos também calculando o limite desejado.


  Faremos isso usando o Teorema Tauberiano, que relaciona o
  comportamente a função de distribuição de uma variável aleatória
  positiva perto do zero com o comportamento de sua transformada de
  Laplace no infinito.  Seguiremos o enunciado do
  Teorema\emph{XIII.5.1} de \cite{fellerv2}.

  O primeiro passo é calcular a transformada da Laplace de
  $\Gamma^\infty(\sigma^x_1-)$. Assim para um $u \geq 0$, podemos
  calcular:
  \begin{align*}
    \phi (u) := \E \left[ e^{-u \Gamma^\infty (\sigma^x_1-)}  \right] =
    \lambda_x \left( \lambda_x + uc + u \sum_{y \neq x}
      \frac{\lambda_x \gamma_x}{1 + u\gamma_x}  \right)^{-1}
  \end{align*}

  Por enquanto vamos nos concentrar no caso $c > 0$. Nesse caso
  teremos que para $u, v > 0$:
  \begin{align*}
    \frac{\phi(uv)}{\phi (u)} &= \frac{\lambda_x + uc + \sum_{y \neq
        x} \frac{u \lambda_x\gamma_x}{1 + u \gamma_x}} {\lambda_x + u
      v c + \sum_{y \neq x} \frac{u v
        \lambda_x\gamma_x}{1 + u v \gamma_x}} \\
    &= \frac{\frac{\lambda_x}{u} + c + \sum_{y \neq x}
      \frac{\lambda_x\gamma_x}{1 + u \gamma_x}} {\frac{\lambda_x}{u} +
      v c + \sum_{y \neq x} \frac{v
        \lambda_x\gamma_x}{1 + u v \gamma_x}}. \\
  \end{align*}

  Agora note que quando $u$ vai ao infinito, os termos de cada uma das
  somas vai a zero monotonamente com $u$, assim usando o teorema da
  convergência monótona, teremos que:
  \begin{align*}
      \lim_{u \to \infty} \frac{\phi(uv)}{\phi (u)} &= \frac{1}{v}.
  \end{align*}

  Assim verificamos a condição do teorema, e temos que:
  \begin{align*}
    \lim_{t \searrow 0} \frac{P( \Gamma^\infty(\sigma^x_1-) \leq
      t)}{\phi(\frac{1}{t})} = 1
  \end{align*}

  Assim se mostrarmos que $\frac{\phi(\frac{1}{t})}{t}$ converge para
  $\frac{\lambda_x}{c}$, teremos mostrado o nosso resultado.

  \begin{align*}
    \frac{\phi(\frac{1}{t})}{t} &= \frac{1}{t} \lambda_x \left(
      \lambda_x + \frac{c}{t} + \sum_{y \neq x} \frac{1}{t}
      \frac{\lambda_x \gamma_x}{1 + \frac{\gamma_x}{t}} \right)^{-1} \\
    &= \lambda_x \left( t\lambda_x + c + \sum_{y \neq x}
      \frac{\lambda_x \gamma_x}{1 + \frac{\gamma_x}{t}} \right)^{-1}.
  \end{align*}

  Novamente cada termo da soma vai a zero monotonamente quanto $t
  \searrow 0$, assim pelo teorema da convergêcia monótona, teremos que
  $\lim_{t \searrow 0} \frac{\phi(\frac{1}{t})}{t} =
  \frac{\lambda_x}{c}$.

  Agora vamos tratar o caso $c = 0$. Para isso note que nossa
  construção do processo K permite que acoplemos várias versões do
  processo, com $c$ diferentes em um mesmo espaço de
  probabilidade. Basta usar os mesmos processos e Poisson e variáveis
  exponenciais para todos eles. Assim vamos colocar um índice $c$ em
  $\Gamma^y_c$ para denotar qual valor de $c$ estamos trabalhando.

  Dessa forma teremos que para todo $y \in \Nzb$, $\Gamma^y_c$ é
  crescente com $c$, e portanto $P ( \Gamma^\infty_c(\sigma^x_1-) \leq
  t)$ é monótona em $c$.

  Dessa forma teremos que, para todo $c > 0$:
  \begin{align*}
    \liminf_{t \searrow 0} \frac{P ( \Gamma^\infty_0(\sigma^x_1-) \leq
      t)}{t} &\geq \liminf_{t \searrow 0} \frac{P (
      \Gamma^\infty_c(\sigma^x_1-) \leq t)}{t}
    = \frac{\lambda_x}{c}
  \end{align*}

  Assim tomando $c > 0$ cada vez menores, concluíremos que $\frac{P (
    \Gamma^\infty_0(\sigma^x_1-))}{t} \xrightarrow{t \searrow 0}
  \infty$
\end{proof}

\begin{proposicao}
  Vale que:
  \begin{displaymath}
    \lim_{t \searrow 0} \frac{p_{\infty \infty}(t) - 1}{t} = -\infty
  \end{displaymath}
\end{proposicao}
\begin{proof}
  O caso $c = 0$ é trivial pela Proposição
  \ref{prop:naocontinuidade}. Assim vamos supor que $c > 0$.
  \begin{align*}
    \frac{p_{\infty \infty} (t) - 1}{t} &= - \sum_{x \in \Nz}
      \frac{p_{\infty x} (t)}{t}.
  \end{align*}

  Cada termo dessa soma converge à $\frac{\lambda_x}{c}$, e estamos
  supondo que a soma dessa série diverge para $\infty$, de onde
  concluímos que:
  \begin{align*}
    \lim_{t \searrow 0}\frac{p_{\infty \infty} (t) - 1}{t} &= - \infty
  \end{align*}
\end{proof}

\begin{proposicao}
  Para $x \in \Nz$, vale que:
  \begin{displaymath}
    \lim_{t \searrow 0} \frac{p_{x \infty}(t)}{t} = \begin{cases}
      \frac{1}{\gamma_x} & \textrm{ se } c > 0 \\
      0 & \textrm{ se } c = 0 \\
    \end{cases}
  \end{displaymath}
\end{proposicao}
\begin{proof}
  Novamente o caso $c = 0$ é trivial em vista da Proposição
  \ref{prop:naocontinuidade}. Assim vamos supor que $c > 0$.
  \begin{align*}
    \frac{p_{x \infty}}{t} &= \frac{1}{t} \int_0^t P(X^x (t) = \infty | \gamma_x
    T^x_0 = s) \frac{1}{\gamma_x}e^{-\frac{s}{\gamma_x}} ds\\
    &= \frac{1}{t} \int_0^t p_{\infty \infty} (t-s)
    \frac{1}{\gamma_x}e^{-\frac{s}{\gamma_x}} ds\\
    &= \frac{e^{-t}}{\gamma_x} \frac{1}{t} \int_0^t p_{\infty \infty} (s)
    e^{\frac{s}{\gamma_x}} ds\\
  \end{align*}

  Como $c>0$, pela Proposição \ref{prop:continuidade}, $p_{\infty
    \infty} (t) \xrightarrow{t \searrow 0} 1$. De onde concluímos que:
   \begin{align*}
    \frac{p_{x \infty}(t)}{t} \xrightarrow{t \searrow 0} 
    \frac{1}{\gamma_x}
  \end{align*}
\end{proof}

%% ------------------------------------------------------------------------- %%

\section{Medida invariante}
\label{sec:invariante}

O objetivo dessa seção é mostrar que a seguinte distrubuição de
probabilidade é uma medida invariante para o processo K.

\begin{equation}
  \label{eq:invariante}
  \pi(x) = \begin{cases}
    \frac{\lambda_x \gamma_x}{c + \sum_{y \in \Nz} \lambda_y \gamma_y}
    & \textrm{ se } x \in \Nz \\
    \frac{c}{c + \sum_{y \in \Nz} \lambda_y \gamma_y}
    & \textrm{ se } x = \infty \\
  \end{cases}
\end{equation}

\begin{proposicao}
  O processo K tem uma única medida invariante.
\end{proposicao}
\begin{proof}
  
\end{proof}


\begin{proposicao}
  A probabilidade $\pi$ definida em \ref{eq:invariante} é a medida
  invariante do processo K, caso $c > 0$.
\end{proposicao}
\begin{proof}
  
\end{proof}

\begin{proposicao}
  A probabilidade $\pi$ definida em \ref{eq:invariante} é a medida
  invariante do processo K, caso $c = 0$.
\end{proposicao}
\begin{proof}
  
\end{proof}


%%% Local Variables: 
%%% TeX-master: "tese"
%%% End: 
