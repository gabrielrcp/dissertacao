%% ------------------------------------------------------------------------- %%
\chapter{Propriedades básicas do Processo K}
\label{cap:propriedades}

Nesse capítulo estabeleceremos resultados que serão ferramentas
básicas para trabalhar com o Processo K durante o resto da
dissertação.

Apresentaremos uma maneira de aproximar o Processo K por processos
Markovianos de salto, bem como provaremos que o processo K é
fortemente Markoviano.

%% ------------------------------------------------------------------------- %%

\section{Observações gerais}
\label{sec:observacoes}

Nessa seção apontaremos várias observações sobre o Processo K, que
serão usadas à exaustão em seções subsequentes.

\begin{proposicao}
  \label{prop:gamma-crescente}
  $\Gamma$ é \qc estritamente crescente.
\end{proposicao}
\begin{proof}
  Como estamos supondo que $\sum_{x \in \Nz} \lambda_x = \infty$,
  então o conjunto das marcas dos processos de Poisson é \qc densa em
  $\R^+$. Em outras palavras, quase certamente, para todo $0 \leq s <
  t$, teremos que existe um $x \in \Nz$ e $i \geq 1$ tal que $s <
  \sigma^x_i < t$. Dessa forma, usando a definição de $\Gamma$, temos
  que:
  \begin{displaymath}
    \Gamma(t) - \Gamma(s) \geq \gamma_x T^x_i > 0.
    \qedhere
  \end{displaymath}
\end{proof}

\begin{proposicao}
  \label{prop:gamma-finita}
  Com probabilidade $1$, $\Gamma(t)$ é finito para todo $t \geq 0$.
\end{proposicao}
\begin{proof}
  Usando o Teorema da Convergência Monótona, podemos calcular:
  \begin{displaymath}
    \E \left[\Gamma^y(t) \right] = \gamma_y + t \sum_{x \in \Nz}
    \lambda_x \gamma_x + ct < \infty
  \end{displaymath}
  Assim para cada $t \geq 0$ fixo, temos que $\Gamma(t)$ é quase
  certamente finito. Tomando uma sequência $t_n \to \infty$ quando $n
  \to \infty$, temos que $P(\cap_{n = 1}^{\infty} \{ \Gamma^y (t_n)
  < \infty \} \cap \{ \Gamma \textrm{ é crescente}\}) = 1$. Assim para
  realizações dentro desse evento, temos que $\Gamma(t_n) < \infty$
  para todo $n$, e assim para cada $t \geq 0$ arbitrário podemos
  escolher um $n$ tal que $t_n > t$. Dessa maneira $\Gamma(t) <
  \Gamma(t_n) < \infty$.
\end{proof}


\begin{proposicao}
  \label{prop:gamma-cadlag}
  A função $\Gamma$ é quase certamente càdlàg, isso é, contínua à
  direita e tem limites à esquerda.
\end{proposicao}
\begin{proof}

  Fixemos uma realização de $\Gamma$ onde ela seja crescente e
  limitada em compactos. Vamos pedir ainda que cada processo de
  Poisson $N_x$, $x \in \Nz$, seja localmente finito. Tais realizações
  têm probabilidade $1$.

  O fato de $\Gamma$ ser crescente e limitada em compactos já
  estabelece diretamente a existência dos limites à direita e à
  esquerda. Agora resta mostrar a continuidade à direita.

  Para isso fixemos um $t \geq 0$ arbitrário. Para $s > 0$ temos:

  \begin{equation*}
    \Gamma(t+s) - \Gamma(t) = 
    c s + 
    \sum_{x \in \Nz} \sum_{i = N_x(t)+1}^{N_x(t+s)} \gamma_x T^x_i.
  \end{equation*}

  Para cada $x$, como $N_x$ é localmente finito, vale que $N_x(t+s) =
  N_x(t)$ para $s$ pequeno o suficiente. Dessa forma cada termo da
  série em $x$ converge à zero. Encarando essa soma como uma
  integral sobre uma medida de contagem e usando o Teorema da
  Convergência Monótona, concluímos que:
  \begin{displaymath}
    \lim_{s \searrow 0} \Gamma(t+s) - \Gamma(t) = 0.
    \qedhere
  \end{displaymath}
\end{proof}

\begin{proposicao}
  \label{prop:gamma-dist-continua}
  Para todo $t > 0$, $i \geq 1$ e $x \in \Nz$ vale que $\Gamma(t)$,
  $\Gamma(\sigma^x_i)$ e $\Gamma(\sigma^x_i-)$ são variáveis
  aleatórias contínuas.
\end{proposicao}
\begin{proof}

  Primeiramente fixemos um $t > 0$ e mostraremos que $\Gamma(t)$ é uma
  variável aleatória contínua.

  Fixemos um boreliano $A$ arbitrário que tenha medida de Lebesgue
  zero. Vamos mostrar que $P(\Gamma(t) \in A) = 0$, de onde seguirá
  através do Teorema de Radon-Nikodym que $\Gamma(t)$ tem uma
  distribuição contínua.


  Para cada $n \in \Nz$, vale que:
  \begin{displaymath}
    P\left(\sum_{i = 1}^n \gamma_x T_i^x \in A \right) = 0,
  \end{displaymath}
  já que as $T_i^x$ são variáveis aleatórias contínuas e
  independentes.

  Dessa forma vale que:
  \begin{align*}
    P\left(\sum_{i = 1}^{N^x(t)} \gamma_x T_i^x \in A \right) 
    &= \sum_{n = 0}^{\infty}
    P\left(\sum_{i = 1}^n \gamma_x T_i^x \in A \right)
    \frac{e^{-\lambda_x t} (\lambda_x t)^n}{n!}\\
    &= \ind\{ 0 \in A\} e^{-\lambda_x t} 
  \end{align*}

  Assim se denotarmos por $\nu$ a distribuição de $\sum_{x \geq 2}
  \sum_{i = 1}^{N^x(s)} \gamma_x T_i^x$ e denotarmos por $A+s$ o
  conjunto $A$ transladado por $s$, teremos que:
  \begin{align*}
    P\left(\sum_{x \in \Nz}\sum_{i = 1}^{N^x(t)} \gamma_x T_i^x \in A  \right) 
    &= \int \nu(d s) P\left(\sum_{i = 1}^{N^1(t)} \gamma_1 T_i^1 \in A - s\right).
  \end{align*}

  Observemos que $0 \in A - s$ se e somente se $s \in A$, dessa forma
  a expressão acima vale $\nu(A) e^{-\lambda_1 t}$.

  Utilizando indução, concluímos que para todo $n \in \Nz$, vale que:
  \begin{align*}
    P\left(\sum_{x \in \Nz}\sum_{i = 1}^{N^x(t)} \gamma_x T_i^x \in A \right) 
    &= \exp\left\{ -t \sum_{x = 1}^n \lambda_x \right\}
    P\left(\sum_{x > n} \sum_{i = 1}^{N^x(t)} \gamma_x T_i^x \in A \right) \\
    &\leq \exp\left\{ -t \sum_{x = 1}^n \lambda_x \right\}.
  \end{align*}

  Como $\sum_{x \in \Nz} \lambda_x = \infty$, tomando valores de $n$
  cada vez maiores, concluímos que:
  \begin{align*}
    P\left(\sum_{x \in \Nz}\sum_{i = 1}^{N^x(t)} \gamma_x T_i^x \in A \right) 
    &= 0.
  \end{align*}

  Observando a definição de $\Gamma$, concluímos a primeira
  parte dessa proposição:
  \begin{displaymath}
    P\left( \Gamma(t) \in A \right) = 0.
  \end{displaymath}


  Fixado $i \geq 1$ e $x \in \Nz$, se denotarmos por $g^x_i$ a
  densidade de uma distribuição gamma de parâmetros $\lambda_x$ e $i$,
  que é a lei de $\sigma^x_i$, teremos que:
  \begin{align*}
    P \left(
      \sum_{z \neq x} \sum_{j = 1}^{N^z(\sigma^x_i)} \gamma_z T^z_j
      + c \sigma^x_i
      \in A
    \right)
    &= \int_0^\infty
     P \left(
      \sum_{z \neq x} \sum_{j = 1}^{N^z(s)} \gamma_z T^z_j
      \in A - c s
    \right)
    g_i^x (s) d s.
  \end{align*}

  Usando argumentos análogos aos que usamos para provar que
  $\Gamma(t)$ é uma variável contínua, concluímos que a probabilidade
  que está sendo integrada vale zero, e portanto a integral também
  vale zero.

  Por construção, observamos que:
  \begin{align*}
    \Gamma^y_c(\sigma^x_i)  &=
     \gamma_y T_0 + \sum_{j = 1}^{i} \gamma_x
    T_j^x + \sum_{z \neq x} \sum_{j = 1}^{N^z(\sigma^x_i)} \gamma_z
    T^z_j + c \sigma^x_i,\\
    \Gamma^y_c(\sigma^x_i -)  &=
     \gamma_y T_0 + \sum_{j = 1}^{i-1} \gamma_x
    T_j^x + \sum_{z \neq x} \sum_{j = 1}^{N^z(\sigma_i^x)} \gamma_z
    T^z_j + c \sigma_i^x.
  \end{align*}

  Dessa forma tanto $\Gamma(\sigma^x_i)$ quanto $\Gamma(\sigma^x_i-)$
  são somas de variáveis aleatórias contínuas e independentes, sendo
  portanto também contínuas.
\end{proof}


\begin{proposicao}
  \label{prop:reinicia-infinito}
  Para todo $y \in \Nz$ e $t \geq 0$, vale que
  $X^y(t + \gamma_y T_0^y) = X^\infty(t)$.
\end{proposicao}
\begin{proof}
  Fixemos um $y \in \Nz$ e observemos que $\Gamma^y(s) =
  \Gamma^\infty(s) + \gamma_yT^y_0$ para todo $s \geq 0$.

  Fixada uma realização do processo, suponha que $X^\infty(t) = x \in
  \Nz$; assim existe um $i \geq 1$ tal que $\Gamma^\infty(\sigma_i^x-)
  \leq t < \Gamma^\infty(\sigma_i^x)$. Dessa forma, somando $\gamma_y
  T^y_0$ aos termos dessa desigualdade, temos que
  $\Gamma^y(\sigma_i^x-) \leq t+\gamma_yT_0^y <
  \Gamma^y(\sigma_i^x)$. E portanto $X^y(t+\gamma_y T_0^y) = x$.

  Com raciocínio análogo, chegamos que se $X^y(t+\gamma_y T_0^y) = x
  \in \Nz$, então $X^\infty(t) = x$. De onde concluímos a igualdade
  desejada.
\end{proof}


\begin{proposicao}
  \label{prop:proc-cadlag}
  O Processo K é quase certamente càdlàg.
\end{proposicao}
\begin{proof}

  Iremos mostrar que o processo iniciado no $\infty$ é càdlàg, o que
  irá mostrar que o processo é càdlàg para qualquer condição inicial,
  visto que, iniciando em $y$, iremos permanecer em $y$ até $\gamma_y
  T^y_0$, e depois continuaremos como uma cópia do processo iniciado
  no $\infty$.

  Fixemos uma realização do processo, $T > 0$ e $\epsilon > 0$
  arbitrários. Seguindo \cite{billingsley:99}, vamos mostrar que
  existem $0 = t_0 < t_1 < \ldots < t_N = T$ tais que $w[t_{i-1}, t_i)
  < \epsilon$ para todo $i = 1, \ldots, N$. Onde $w(A) = \sup_{t, s
    \in A} d(X^\infty(t), X^\infty(s))$ para $A \subseteq
  \R$. Lembrando que estamos trabalhando com a métrica
  \eqref{eq:metrica} sobre $\Nzb$.

  Tomemos um $m \in \Nz$ tal que $\diam\{x \in \Nzb: x > m \} =
  \frac{1}{m+1} < \epsilon$, onde $\diam(A)$ é o diâmetro do conjunto
  $A$.

  Agora tomemos $S_1 < \ldots < S_M$ uma ordenação dos conjunto $\{
  \sigma^x_i: x \leq m, \: i \geq 1, \: \Gamma(\sigma^x_i) \leq
  T\}$. Finalmente fixemos $N = 2M+1$ se $\Gamma(S_M) < T$ e $N = 2M$
  se $\Gamma(S_M) = T$. Tomemos $t_0 = 0$, $t_N = T$ e para $i=1,\ldots, M$:
  \begin{align*}
    t_{2i-1} &= \Gamma(S_i-)\\
    t_{2i} &= \Gamma(S_i).\\
  \end{align*}

  Se $t \in [\Gamma(S_i-), \Gamma(S_i))$, temos que $X(t)$ é
  constante, enquanto que se $t \in [\Gamma(S_{i-1}),
  \Gamma(S_{i}-))$, temos que $X(t) > m$. Assim a variação nesse
  intervalo é menor ou igual à $\frac{1}{m} < \epsilon$. O mesmo
  ocorre nos intervalos $[t_0, t_1)$ e $[t_{N-1}, t_N)$.
\end{proof}

\begin{proposicao}
  \label{prop:proc-descontinuidades}
  $(X(t))_{t\geq 0}$ é \qc contínuo fora de $\{ \Gamma(0),
  \Gamma(\sigma_i^x-), \Gamma(\sigma_i^x): x \in \Nz\}$.
\end{proposicao}
\begin{proof}
  Tomemos $t$ um ponto de descontinuidade de $X$. Como $X$ é càdlàg,
  então existe um $\epsilon > 0$ tal que $d(X(t), X(t-)) > \epsilon$.
  Fixemos um $T > t$ e tomemos $0 = t_0 < t_1 < \ldots < t_N = T$ como na
  demonstração da Proposição \ref{prop:proc-cadlag}. Se $t \not\in
  \{t_0, \ldots, t_N\}$, então existe um $i$ tal que $t \in (t_i,
  t_{i+1})$. Assim temos que $w[t_i, t_{i+1}) \geq d(X(t), X(t-)) >
  \epsilon$, o que contraria a nossa escolha de $\{t_0, \ldots, t_N\}$.

  Assim temos que $t = t_i$ para algum $i$. Como os $t_i$'s sempre
  foram escolhidos do conjunto proposto, temos que eles são os únicos
  candidatos possíveis para pontos de descontinuidade.
\end{proof}

\begin{corolario}
  \label{cor:continuidades-processo}
  Qualquer $t > 0$ é quase certamente um ponto de continuidade do
  processo K.
\end{corolario}
\begin{proof}
  Mostramos que o conjunto dos pontos de pontos de descontinuidade do
  processo está contido em $\{ \Gamma(0), \Gamma(\sigma_i^x-),
  \Gamma(\sigma_i^x): x \in \Nz\}$.

  Observando a Proposição \ref{prop:gamma-dist-continua}, concluímos
  que todo $t > 0$ é um ponto de continuidade do processo.
\end{proof}


%% ------------------------------------------------------------------------- %%

\section{Aproximações}
\label{sec:aproximacoes}

Nessa seção vamos introduzir o que chamaremos de processos truncados,
que serão processos markovianos de saltos que irão convergir para o
Processo K original.

Primeiramente, para $n \in \Nz$ e $y \in \{1, \ldots, n, \infty\}$,
vamos definir:
\begin{displaymath}
  \Gamma^{(n)} (t) := \Gamma^{y,(n)}_c (t) = \gamma_y T_0
  + \sum_{x =1}^{n} \sum_{i = 1}^{N_x(t)}
  \gamma_x T_i^x
  + ct.
\end{displaymath}

O processo truncado em $n \in \Nz$, com estado inicial $y \in \{1,
\ldots, n, \infty\}$ será:
\begin{displaymath}
  X^{(n)}(t) = X^{y,(n)}_c(t) = \begin{cases}
    y, & \textrm{ se }  t < \gamma_y T_0 \\
    x, & \textrm{ se } \Gamma^{y,(n)}_c(\sigma_i^x-) \leq t <
    \Gamma^{y,(n)}_c(\sigma^x_i)
    \text{ para algum } i \\
    \infty, & \textrm{ caso contrário.}
  \end{cases}
\end{displaymath}

\begin{proposicao}
  O processo $X^{(n)}$ é Càdlàg e Markoviano.
\end{proposicao}
\begin{proof}
  Quando truncamos o processo K, ele se comporta de maneira análoga ao
  caso analisado na Seção \ref{sec:visualizacao}, assim processo
  $X^{(n)}$ é um processo Markoviano de saltos, construído para ser
  contínuo à direita.
\end{proof}

Construímos $\{(X^{(n)}(t))_t\}_{n \in \Nz}$ e $(X(t))_t$ num mesmo
espaço de probabilidade. Assim é natural perguntar se $(X^{(n)}(t))_t$
converge para $(X(t))_t$ de alguma forma.

\begin{teorema}
  \label{teo:convergencia}
  $X^{\infty, (n)} (\bullet)$ converge para $X^\infty(\bullet)$ \qc na
  métrica de Skorohod quando $n \to \infty$.
\end{teorema}

A métrica de Skorohod é uma métrica sobre o espaço das trajetórias
càdlàg que permite pequenas distorções temporais. Como
referência recomendamos \cite{billingsley:99} e \cite{ethier:86}.

\begin{proof}
  Essa demonstração foi adaptada diretamente do \emph{Lema 3.11} de
  \cite{fontes:08}.

  Mostraremos que o item (c) da Proposição 5.3 do Capítulo 3 de
  \cite{ethier:86} vale quase certamente.

  Como o processo sempre se inicia no $\infty$, não vamos mais
  carregar esse índice na notação.

  Fixemos uma realização do processo e um $T > 0$. Para cada natural
  $m$, considere $\delta_m := \diam\{ x \in \Nzb: x > m \} =
  \frac{1}{m+1}$. Tome ainda $0 = S_0^m < S_1^m < \ldots $ uma
  ordenação de $\{0\}\cup\{ \sigma^x_i : x \leq m, i \geq
  1\}$. Considere ainda, para cada $n > m$:
  \begin{displaymath}
    L^m_n := \min \left\{ i \geq 1: \Gamma^{(n)}(S^m_i) \geq T \right\}.
  \end{displaymath}

  O conjunto $\{\sigma_i^x: x > m, i\geq 1\}$ é denso; assim, para $n$
  suficientemente grande, $\Gamma^{(n)}(S^m_{i+1}-) >
  \Gamma^{(n)}(S^m_i)$ para todo $i < L^m_n$.

  Para esses valores de $n$, definimos $\lambda_n^m: [0, \infty) \to
  \R^+$ da seguinte forma:
  \begin{displaymath}
    \lambda_n^m(t) = \begin{cases}
      \Gamma(S_i^m) + \frac{\Gamma(S_{i+1}^m-) - \Gamma(S_i^m)}
      {\Gamma^{(n)}(S_{i+1}^{(m)} -) - \Gamma^{(n)}(S_i^m)}
      \left[t - \Gamma^{(n)}(S_i^m)\right]
      & \textrm{ se }
      \Gamma^{(n)}(S_i^m) \leq t \leq \Gamma^{(n)}(S_{i+1}^m-) \\
      \Gamma(S_{i+1}^m-) - \Gamma^{(n)}(S_{i+1}^m-) + t
      & \textrm{ se }
      \Gamma^{(n)}(S_{i+1}^m-) \leq t \leq \Gamma^{(n)}(S_{i+1}^m)\\
      \Gamma(S^m_{L_n^m}) - \Gamma^{(n)}(S^m_{L_n^m}) + t
      & \textrm{ se } t > \Gamma^{(n)}(S^m_{L_n^m})
    \end{cases}
  \end{displaymath}

  Para entendermos o que motivou essa definição, observemos que para $i =
  0, \ldots L_n^m$, temos que:
  \begin{align*}
    \lambda_n^m(\Gamma^{(n)}(S_i^m-)) &= \Gamma(S_i^m-)\\
    \lambda_n^m(\Gamma^{(n)}(S_i^m)) &= \Gamma(S_i^m),
  \end{align*}
  enquanto que nos pontos interiores interpolamos $\lambda_n^m$ de
  maneira linear. Depois de $\Gamma^{(n)}(S^m_{L_n^m})$, deixamos
  $\lambda_n^m$ evoluir de maneira linear com coeficiente 1, já que
  isso terá mais importância.

  Como $\Gamma(t) \geq \Gamma^{(n)}(t)$ para todo $n \in \Nz$, temos
  que $\lambda_n^m(t) \geq t$. Usando a linearidade por partes, o fato
  de que $\Gamma(\sigma^x_i) = \Gamma(\sigma_i^x-) + \gamma_x T^x_i$ e
  $\Gamma^{(n)}(S^m_{L_N^m}) \geq T$, teremos que:
  \begin{displaymath}
    \sup_{0 \leq t \leq T} |\lambda_n^m(t) - t| \leq
    \max_{0 \leq i \leq L_n^m} \{ \Gamma(S_i^m) -
    \Gamma^{(n)}(S_i^m)\}.
  \end{displaymath}

  Essa quantidade converge quase certamente a zero se mantivermos o
  $m$ fixo e jogarmos o $n$ para infinito. Assim para cada $m$ existe
  um $n_m$ tal que para $n \geq n_m$ vale que:
  \begin{displaymath}
    \sup_{0 \leq t \leq T} |\lambda_n^m(t) - t| < \delta_m.
  \end{displaymath}

  Podemos tomar a sequência $(n_m)_{m \geq 1}$ de modo que ela seja
  crescente. Agora vamos ``invertê-la'', isso é, para cada n em $\Z
  \cap [n_{i-1}, n_i)$, definimos $m_n = i$. Teremos que:
  \begin{displaymath}
    \sup_{0 \leq t \leq T} |\lambda_n^{m_n}(t) - t| < \delta_{m_n}.
  \end{displaymath}

  Observemos que $m_n$ vai ao infinito quando $n \to \infty$ já que
  $n_m$ é finito para cada $m$ fixado.
  
  Por construção, $t \in [\Gamma^{(n)}(S_{i}^m-),
  \Gamma^{(n)}(S_{i}^m))$ se e somente se $\lambda_n^m(t) \in
  [\Gamma(S_{i}^m-), \Gamma(S_{i}^m))$. Assim, para cada $x \in \{1,
  \ldots, m\}$, temos que $X(\lambda_n^m(t)) = x$ se e somente se
  $X^{(n)}(t) = x$. De onde concluímos que:
  \begin{displaymath}
    \sup_{0 \leq t \leq T} d\left(X(\lambda_n^m(t)), X^{(n)} (t)\right)
    \leq \delta_m.
  \end{displaymath}

  Tomando $\tilde{\lambda}_n = \lambda_n^{m_n}$, concluímos que:
  \begin{align*}
    \sup_{0 \leq t \leq T} |\tilde{\lambda}_n(t) - t|
    &\xrightarrow{n\to\infty} 0 \\
    \sup_{0 \leq t \leq T} d(X(\tilde{\lambda}_n(t)), X^{(n)}(t))
    &\xrightarrow{n\to\infty} 0
    \qedhere
  \end{align*}
\end{proof}

\begin{corolario}
  \label{cor:convergencia}
  Para todo $y \in \Nzb$, todo $T > 0$ e quase toda trajetória do
  processo, existe uma sequência de funções $\lambda^y_n: [0, \infty) \to
  [0, \infty)$ contínuas e bijetoras tais que:
  \begin{align}
    \lambda^y_n(t) &\geq t \notag\\
    \label{eq:convergencia-truncado-uniforme}
    \sup_{t \in [0, T]} |\lambda^y_n(t) - t| &\leq
    \sup_{t \in [0, T]} |\lambda^\infty_n(t) - t|
    \xrightarrow{n \to \infty} 0\\
    \sup_{0 \leq t \leq T} d(X^y(\lambda_n^y(t)), X^{y, (n)}(t)) &\leq
    \sup_{0 \leq t \leq T} d(X^\infty(\lambda_n(t)), X^{\infty, (n)}(t))
    \xrightarrow{n\to\infty} 0. \notag
  \end{align}

  Dessa forma, temos que $X^{y, (n)}$ converge \qc para $X^y$ na
  métrica de Skorohod, sendo essa convergência uniforme em $y$.
\end{corolario}
\begin{proof}
  Fixado um $T > 0$, tomemos $\lambda_n: [0, \infty) \to [0, \infty)$
  como na demonstração do \mbox{Teorema \ref{teo:convergencia}}. Elas
  são funções contínuas crescentes tais que quase certamente:
  \begin{gather*}
    \lambda_n(t) \geq t\\
    \sup_{0 \leq t \leq T} |\lambda_n(t) - t|
    \xrightarrow{n\to\infty} 0 \\
    \sup_{0 \leq t \leq T} d(X^\infty(\lambda_n(t)), X^{\infty, (n)}(t))
    \xrightarrow{n\to\infty} 0.
  \end{gather*}

  Tomemos $\lambda^\infty_n = \lambda_n$ e para $y \in \Nz$ definimos:
  \begin{displaymath}
    \lambda_n^y(t) = \begin{cases}
      t, & \textrm{ se } t < \gamma_y T_0\\
      \gamma_yT_0 + \lambda_n(t - \gamma_y T_0),
      & \textrm{ se } t \geq \gamma_y T_0
    \end{cases}
  \end{displaymath}


  Sabemos pela Proposição \ref{prop:reinicia-infinito} que o processo
  iniciado em um $y \in \Nz$ permanece em $y$ até o instante $\gamma_y
  T_0$. Depois disso ele será uma cópia do processo iniciado no
  $\infty$. Dessa forma forma concluímos
  \eqref{eq:convergencia-truncado-uniforme}.
\end{proof}


%% ------------------------------------------------------------------------- %%

\section{Propriedade de Markov}
\label{sec:prop-markov}

O objetivo dessa seção é provar que o Processo K é Markoviano. Nossa
estratégia será mostrar que a propriedade de Markov se mantém quando
tomamos os limites dos processos truncados introduzidos na Seção
\ref{sec:aproximacoes}.

Note que nem sempre o limite de processos Markovianos é um processo
Markoviano. Para demonstrar que essa propriedade é mantida, vamos
abusar de diversas propriedades da nossa construção, em particular da
convergência dos processos truncados uniformemente na condição
inicial.

Uma dificuldade introduzida ao aceitar pesos não homogêneos é a de que
o processo pode não ser de Feller. Isso acontece porque podem existir
estados $y$ ``grandes'', onde $\gamma_y$ também é grande; assim
iniciar o processo nesses estados é diferente de iniciar no
$\infty$. Mostraremos isso com detalhes na Seção
\ref{sec:prop-feller}.

Vamos contornar esse problema usando o fato que esses estados onde
$\gamma_y$ é grande não são de fato visitados. Essa ideia será
formalizada pela Proposição \ref{prop:gamma-somavel}. Com isso
mostraremos que nosso processo tem propriedades muito parecidas com as
de um processo de Feller.

\begin{proposicao}
  \label{prop:gamma-somavel}
  Seja $V_t := \cup_{s \in [0, t]} \{ X(s) \} \setminus \{\infty\}$ o
  conjunto de estados visitados pelo processo até o instante
  $t$. Então:
  \begin{displaymath}
    \sum_{x \in V_t} \gamma_x < \infty,
  \end{displaymath}
  quase certamente para todo $t > 0$.
\end{proposicao}
\begin{proof}

  Primeiramente fixemos um $t > 0$ arbitrário.

  Denotemos por $V^\prime_t = \{ x \in \Nz: N^x(t) \geq 1 \}$. Quase
  certamente vale que:
  \begin{displaymath}
    \sum_{x \in V^\prime_t} \gamma_x T^x_1 \leq \Gamma(t) < \infty.
  \end{displaymath}

  Se denotarmos por $\nu$ a distribuição de $V^\prime_t$ sobre o
  conjunto das partes de $\Nz$, como $V^\prime_t$ só depende dos
  processos de Poisson $\{N^x: x \in \Nz\}$ e como esses são
  independentes de $\{ T^x_1: x \in \Nz \}$, temos que:
  \begin{align*}
    1 &= P\left(\sum_{x \in V^\prime_t} \gamma_x T^x_1 < \infty
    \right)\\
    &=\int P\left(\sum_{x \in V} \gamma_x T^x_1 < \infty
    \right) \nu(dV)
  \end{align*}

  Portanto, para $\nu$ quase todo $V \subseteq \Nz$ vale que
  $P(\sum_{x \in V} \gamma_x T^x_1 < \infty) = 1$. Fixado um
  $V$ desses, teremos que $\sum_{x \in V}\gamma_x < \infty$, já que
  uma soma de variáveis aleatórias exponenciais independentes é finita
  \qc se e somente se a soma de suas médias convergir.

  Com isso concluímos que para $\nu$ quase todo $V$ vale que $\sum_{x
    \in V}\gamma_x < \infty$, ou seja, $\sum_{x \in
    V^\prime_t}\gamma_x < \infty$ \qc.

  Agora fixemos uma sequência $(t_n)$, com $t_n\nearrow
  \infty$. Com probabilidade $1$ vai valer que $\sum_{x \in
    V^\prime_{t_n}}\gamma_x < \infty$ para todo $n$.

  Notemos que $V^\prime_t \subseteq V^\prime_s$ sempre que $t \leq
  s$. Para qualquer $t > 0$, existe um $n$ tal que $t_n > t$, e
  portanto:
  \begin{displaymath}
    \sum_{x \in V^\prime_t}\gamma_x T^x_1 \leq
    \sum_{x \in V^\prime_{t_n}}\gamma_x T^x_1 < \infty
  \end{displaymath}

  Voltando a trabalhar com $V_t$, notemos que se $s < t$, então $V_s
  \subseteq V_t$. Observemos ainda que, por construção, $V^\prime_t =
  V_{\Gamma(t)}$.

  Agora fixemos uma realização do processo, onde $\Gamma(t) \to
  \infty$ quando $t \to \infty$ e $\sum_{x \in V^\prime_t} \gamma_x <
  \infty$ para todo $t$. Pelo argumento anterior tais realizações têm
  probabilidade $1$. Nelas, para um $t > 0$ arbitrário, vai existir
  um $s > 0$ tal que $\Gamma(s) > t$. Dessa forma:
  \begin{displaymath}
    \sum_{x \in V_t} \gamma_x \leq \sum_{x \in V_{\Gamma(s)}}
    \gamma_x =
    \sum_{x \in V^\prime_s} \gamma_x < \infty.
  \end{displaymath}

  De onde concluímos o que desejávamos.
\end{proof}


\begin{definicao}
  \label{def:semigrupo}
  Denotaremos por $\Psi$ o semigrupo de transição do Processo K, isso
  é, para $t > 0$ e $f: \Nzb \to \R$:
  \begin{gather*}
    \Psi_t f (x) = \E \left[ f(X^x(t)) \right]
  \end{gather*}
\end{definicao}

\begin{proposicao}
  \label{prop:semigrupo-quase-continuo}
  Para $f: \Nzb \to \R$ contínua e $t, s > 0$ fixados arbitrariamente,
  vale que \qc:
  \begin{equation}
    \label{eq:semigrupo-quase-continuo}
    \lim_{n \to \infty} \Psi_s f (X^{(n)}(t)) = \Psi_s f(X(t))
  \end{equation}
\end{proposicao}

\begin{proof}
  Usando o Corolário \ref{cor:continuidades-processo}, temos que $t$ é
  \qc um ponto de continuidade do processo original. Assim, como
  $X^{(n)}$ converge \qc para $X$ na métrica de Skorohod, ele também
  converge pontualmente nos pontos de continuidade de $X$. Portanto
  $X^{(n)}(t) \to X(t)$ \qc quando $n \to \infty$.

  Vamos fixar agora uma dessas realizações e denotar por $y_n =
  X^{(n)}(t)$ e $y = X(t)$. Vamos separar em dois casos.

  O primeiro admite que existe um $n_0$ tal que $y_n = y$ para todo $n >
  n_0$. Nesse caso \eqref{eq:semigrupo-quase-continuo} é evidente.

  O segundo caso é o contrário do primeiro, isso é, para todo $n_0$
  existe um $n > n_0$ tal que $y_n \neq y$. Como $y_n \to y$, então
  concluímos que $y$ não é um ponto isolado; como o único ponto que
  não é isolado em $\Nzb$ é o $\infty$, concluímos que $y = \infty$.

  Vamos ignorar os índices $n$ onde $y_n = \infty$, visto que neles a
  igualdade em \eqref{eq:semigrupo-quase-continuo} é trivial. 

  Tomemos um $r > 0$ tal que $\Gamma^{(1)}(r-) > t$ e $t^\prime =
  \Gamma(r)$. Se um estado $x\in \Nz$ foi visitado por um processo
  truncado $X^{(n)}$ até o instante $t$, então
  $\Gamma^{(n)}(\sigma_1^x-) \leq t$. Como $\Gamma^{(n)}$ é não
  decrescente em $n$, então $\sigma_1^x < r$. Já que $\Gamma(\bullet)$
  é crescente, valerá que $\Gamma(\sigma_1^x-) < \Gamma(r) =
  t^\prime$. Portanto se $x \in \Nz$ foi visitado por um processo
  truncado, então $x \in V_{t^\prime}$.

  Dessa forma, usando a Proposição \ref{prop:gamma-somavel} e o fato
  de que $y_n \to \infty$, concluímos que $\gamma_{y_n} \to 0$ quando
  $n \to \infty$.

  Se tomarmos um $n$ suficientemente grande teremos que $\gamma_{y_n}
  T_0 < s$. Dessa forma usando a Proposição
  \ref{prop:reinicia-infinito} teremos que:
  \begin{align*}
    \Psi_s f (\infty) - \Psi_s f (y_n) &=
    \E\left[ f(X^\infty(s)) - f(X^{y_n}(s)) \right] \\
    &= \E\left[ f(X^\infty(s)) -f (X^\infty(s - \gamma_{y_n}T_0)) \right] \\
  \end{align*}

  Pelo Corolário \ref{cor:continuidades-processo}, $s$ é \qc um ponto
  de continuidade de $X^\infty$ e, como $\gamma_{y_n} T_0$ converge à
  zero, temos que a quantidade dentro da esperança acima converge
  quase certamente à zero. Como $f$ é contínua - e, portanto, limitada
  - o Teorema da Convergência Dominada nos garante que a esperança
  também converge à zero.
\end{proof}


\begin{teorema}
  \label{teo:proc_markov}
  O Processo K é um processo Markoviano.
\end{teorema}

\begin{proof}
  Fixamos $m \geq 1$, $t_1 < t_2 < \ldots < t_{m+1}$ e $f_1, \ldots,
  f_{m+1} : \Nzb \to \R$ funções contínuas.

  Como $X^{(n)}$ é Markoviano, se denotarmos por $\Psi^n$ o semigrupo
  de transição do processo truncado, teremos que:
  \begin{gather}
    \label{eq:markov-truncado}
    \E \left[
      f_{1}(X^{(n)}(t_{1})) 
      \ldots
      f_{m}(X^{(n)}(t_{m})) 
      f_{m+1}(X^{(n)}(t_{m+1})) 
    \right] \\
    = \E \left[
      f_{1}(X^{(n)}(t_{1})) 
      \ldots
      f_{m}(X^{(n)}(t_{m})) 
      \Psi^n_{t_{m+1} - t_{m}} f_{m+1} (X^{(n)}(t_{m})) 
    \right]\notag
  \end{gather}

  O Corolário \ref{cor:continuidades-processo} nos garante que \qc
  $t_1, \ldots, t_{m+1}$ são pontos de continuidade do processo e,
  como a convergência na métrica de Skorohod garante a convergência
  pontual nos pontos de continuidade, temos que $f_i(X^{(n)}(t_i)) \to
  f_i(X(t_i))$ \qc já que $f_i$ é contínua, $i = 1, \ldots, m+1$.

  Temos ainda que $f_1, \ldots f_{m+1}$ são limitadas porque são
  funções contínuas num espaço compacto. Assim, usando o Teorema da
  Convergência Dominada, teremos que a parte da esquerda de
  \eqref{eq:markov-truncado} converge para:
  \begin{displaymath}
    \E \left[
      f_{1}(X(t_{1})) 
      \ldots
      f_{m}(X(t_{m})) 
      f_{m+1}(X(t_{m+1})) 
    \right].
  \end{displaymath}

  Agora vamos estimar a parte da direita por:
  \begin{equation}
    \label{eq:markov-est-eps}
    \E \left[
      f_{1}(X^{(n)}(t_{1})) 
      \ldots
      f_{m}(X^{(n)}(t_{m})) 
      \Psi_{t_{m+1} - t_{m}} f_{m+1} (X^{(n)}(t_{m})) 
    \right] + \epsilon_n.
  \end{equation}

  Usando a Proposição \ref{prop:semigrupo-quase-continuo}, temos que a
  parte da esquerda de \eqref{eq:markov-est-eps} converge para:
  \begin{displaymath}
    \E \left[
      f_{1}(X(t_{1})) 
      \ldots
      f_{m}(X(t_{m})) 
      \Psi_{t_{m+1} - t_{m}} f_{m+1} (X(t_{m})) 
    \right].
  \end{displaymath}

  Resta mostrar que $\epsilon_n \to 0$. Para simplificar a notação,
  vamos denotar por $s = t_{m+1} - t_m$, $t = t_m$ e $g = f_{m+1}$.
  Usando o fato de $f_1, \ldots, f_m$ serem limitadas, teremos que:
  \begin{displaymath}
    |\epsilon_n| \leq  C
    \left\lvert \E \left[
        \Psi_{s}^n g (X^{(n)}(t)) -
        \Psi_{s} g (X^{(n)}(t))
      \right]
    \right\rvert,
  \end{displaymath}
  onde $C$ é uma constante positiva.

  Como a parte interna dessa esperança pode ser dominada por $2\lVert
  g \rVert$, se mostrarmos que ela converge quase certamente para
  zero, seguirá do Teorema da Convergência Dominada que a esperança também
  converge à zero.

  Com o mesmo argumento que usamos na demonstração da Proposição
  \ref{prop:semigrupo-quase-continuo}, temos que existe um $t^{\prime}
  > 0$ tal que $X^{(n)}(t) \in V_{t^\prime}$ para todo $n$. Dessa forma:
  \begin{align*}
    \left\lvert \Psi_{s}^n g (X^{(n)}(t)) - \Psi_{s} g (X^{(n)}(t))
    \right\rvert
    &\leq \sup_{y \in V_{t^\prime}}  \left\lvert \Psi_{s}^n g (y) - \Psi_{s} g (y)
    \right\rvert
  \end{align*}

  A Proposição \ref{prop:gamma-somavel} nos garante que, para quase
  todo $V_{t^\prime}$, a condição da Proposição
  \ref{prop:convergencia-semigrupo} é satisfeita. Portanto a
  quantidade acima vai à zero \qc.

  Dessa forma podemos concluir que:
  \begin{displaymath}
    \E \left[
      f_{1}(X(t_{1})) 
      \ldots
      f_{m}(X(t_{m})) 
      f_{m+1}(X(t_{m+1})) 
    \right] \notag\\
    = \E \left[
      f_{1}(X(t_{1})) 
      \ldots
      f_{m}(X(t_{m})) 
      \Psi_{t_{m+1} - t_{m}} f_{m+1} (X(t_{m})) 
    \right].
    \qedhere
  \end{displaymath}
\end{proof}


\begin{proposicao}
  \label{prop:convergencia-semigrupo}
  Seja $A \subseteq \Nz$ tal que
  \begin{displaymath}
    \lim_{n \to \infty} \sup\{ \gamma_x: x \in  A \textrm{ e } x > n\} = 0,
  \end{displaymath}
  sob a convenção de que $\sup \emptyset = 0$. Tomemos ainda $f: \Nzb
  \to \R$ contínua e $t > 0$. Nessas condições:
  \begin{equation}
    \label{eq:convergencia-semigrupo}
    \sup_{y \in A} | \Psi^n_t f (y) - \Psi_t f(y) |
    \xrightarrow{n\to\infty} 0,
  \end{equation}
  onde $\Psi^n$ denota o semigrupo de transição de $X^{(n)}$.
\end{proposicao}

\begin{proof}

  Mostremos primeiramente que cada ``termo'' de
  \eqref{eq:convergencia-semigrupo} converge para zero, ou seja,
  mostremos que $\Psi^n_t f (y) \to \Psi_t f(y)$ quando $n \to \infty$
  para qualquer $y \in A$ fixado.

  \begin{align}
    \label{eq:semigrupo-convergencia-pontual}
    \Psi^n_t f (y) -\Psi_t f(y) &=
    \E \left[
      f(X^{y, (n)}(t)) - f(X^y(t))
    \right]
  \end{align}

  O Corolário \ref{cor:continuidades-processo} nos garante que $t$ é
  \qc um ponto de continuidade de $X^y$, assim $X^{y, (n)}(t) \to
  X^y(t)$ \qc quando $n \to \infty$ e, como $f$ é contínua, concluímos
  que $f(X^{y, (n)}(t)) \to f(X^y(t))$ \qc.

  Mas $f$ também é limitada e assim considerando o Teorema da
  Convergência Monótona concluímos que
  \eqref{eq:semigrupo-convergencia-pontual} converge à zero quando $n
  \to \infty$.

  Com isso o caso em que $A$ é finito é trivial; portanto vamos supor
  que $A$ é infinito.

  Tomemos $\lambda_n^y$ como no Corolário \ref{cor:convergencia} para
  um $T > t$ qualquer. Vale que:
  \begin{align}
    \label{eq:esperancas-markov}
    \sup_{y \in A} | \Psi^n_t f (y) - \Psi_t f(y) | 
    &= \sup_{y \in A} \left\lvert \E \left[f(X^{y, (n)}(t)) -
        f(X^y(t)) \right]\right\rvert \\
    &\leq \sup_{y\in A}\left\lvert \E \left[f(X^{y, (n)}(t)) -
        f(X^y(\lambda_n^y(t))) \right]
    \right\rvert \notag \\
    &+ \sup_{y\in A}\left\lvert \E \left[f(X^y(\lambda_n^y(t))) -
        f(X^y(t)) \right] \right\rvert. \notag
  \end{align}

  Vamos tratar os dois termos dessa soma separadamente.

  Por \eqref{eq:convergencia-truncado-uniforme}, temos que para todo
  $y$, quase certamente:
  \begin{equation}
    \label{eq:dist-uniforme}
    d(X^{y, (n)}(t), X^y(\lambda_n^y(t)) \leq
    \sup_{s \in [0, T]} d(X^{\infty, (n)}(s), X^\infty(\lambda_n^\infty(s))
    \xrightarrow{n\to\infty} 0.
  \end{equation}
  
  Como $f$ é uma função contínua em um espaço compacto,  ela é
  uniformemente contínua. Isso é, para todo $\epsilon > 0$, existe um
  $\delta_\epsilon > 0$ tal que se $d(x, y) < \delta_\epsilon$ então
  $|f(x)-f(y)| < \epsilon$.

  Dessa forma, fixado um $\epsilon > 0$ arbitrário, teremos que:
  \begin{align*}
    \sup_{y\in A}& \left\lvert \E \left[ f(X^{y,(n)}(t)) -
        f(X^{y}(\lambda_n^y(t))) \right]
    \right\rvert\\
    &\leq \sup_{y\in A} \left\lvert \E \left[ \left(f(X^{y,(n)}(t)) -
          f(X^{y}(\lambda_n^y(t))) \right) \ind\{ d(X^{y, (n)}(t),
        X^y(\lambda_n^y(t)) < \delta_\epsilon \} \right]
    \right\rvert \\
    &+ 2\Vert f \Vert \sup_{y \in A} P\left( d(X^{y, (n)}(t),
      X^y(\lambda_n^y(t)) \geq
      \delta_\epsilon \right) \\
    &\leq \epsilon + 2 \Vert f \Vert P \left( \sup_{s \in [0,
        T]}d(X^{\infty, (n)}(s), X^\infty(\lambda_n^\infty(s)) \geq
      \delta_\epsilon \right).
  \end{align*}
  Essa última quantidade converge para $\epsilon$ quando $n\to \infty$
  devido à \eqref{eq:dist-uniforme}. Como $\epsilon$ foi escolhido
  arbitrariamente, concluímos que o primeiro termo de
  \eqref{eq:esperancas-markov} converge à zero.

  Para calcular o segundo termo, notemos que aquela quantidade, para
  cada $y \in A$ fixado vai à zero, já que $t$ é \qc um ponto de
  continuidade do processo e $\lambda_n^y(t) \to t$ quando
  $n\to\infty$. Dessa forma existe uma sequência $(k_n)$ tal que $k_n
  \xrightarrow{n\to\infty} \infty$ onde:
  \begin{displaymath}
      \max_{\stackrel{y \leq k_n}{y\in A}} \left\lvert
      \E \left[f(X^y(\lambda_n^y(t))) - f(X^y(t)) \right]
    \right\rvert \xrightarrow{n\to \infty} 0.
  \end{displaymath}

  Agora vamos tomar $\delta_n = \sqrt{\sup\{ \gamma_x: x \in A
    \textrm{ e } x > k_n\}}$. Por hipótese, temos que $\delta_n \to
  0$. Notemos ainda que $\sup \{ P(\gamma_y T_0 > \delta_n) : y \in A, y
  > k_n \} = e^{-\frac{1}{\delta_n}} \to 0$ quanto $n \to \infty$.

  Como $\delta_n \to 0$, vamos apenas considerar $n$ suficientemente
  grande, de forma que $\delta_n < t \leq \lambda_n^y(t)$. A segunda
  desigualdade vale de acordo com
  \eqref{eq:convergencia-truncado-uniforme}. Usando a Proposição
  \ref{prop:reinicia-infinito} e separando nos casos $\gamma_yT_0 <
  \delta_n$ e $\gamma_yT_0 \geq \delta_n$, teremos que:
  \begin{align}
    \label{eq:markov-quase-la}
    \sup_{\stackrel{y > k_n}{y \in A}}& \left\lvert \E \left[f(X^y(\lambda_n^y(t))) -
        f(X^y(t)) \right]
    \right\rvert \notag \\
    &\leq \sup_{\stackrel{y > k_n}{y \in A}} \E \left\lvert
      \left[f(X^\infty(\lambda_n^y(t)-\gamma_yT_0)) -
        f(X^\infty(t-\gamma_yT_0)) \right] \ind\{ \gamma_yT_0 <
      \delta_n \}
    \right\rvert\\
    &+ 2 \Vert f \Vert \sup_{\stackrel{y > k_n}{y \in A}} P(\gamma_y T_0 > \delta_n).
    \notag
  \end{align}

  Nós escolhemos $\delta_n$ de forma que o segundo termo dessa soma vá
  à zero quando $n \to \infty$. Podemos dominar o primeiro termo por:
  \begin{displaymath}
     \E \left[ \sup_{\stackrel{y > k_n}{y\in A}}
      \sup_{0 \leq s \leq \delta_n} \left\lvert
        f(X^\infty(\lambda_n^y(t)-s)) -
        f(X^\infty(t-s))
    \right\rvert \right].
  \end{displaymath}

  A variável sobre a qual estamos tomando esperança é dominada por
  $2\Vert f \Vert$. Portando se mostrarmos que ela converge \qc para
  zero, valerá que sua esperança também vai a zero.

  Fixado um $\epsilon > 0$, tomemos $\epsilon^\prime > 0$ tal que
  $d(x, y) < \epsilon^\prime$ implique que $|f(x) - f(y)| < \epsilon$;
  isso existe por causa da continuidade uniforme de $f$.
 
  Como $t$ é \qc um ponto de continuidade de $X^\infty$, existe um
  $\epsilon^{\prime\prime}$ tal que se $|s| <
  \epsilon^{\prime\prime}$, então $d(X^\infty_t, X^\infty_{t+s}) <
  \frac{\epsilon^{\prime}}{2}$.

  Assim se tomarmos $n_0$ tal que para todo $n > n_0$, $\delta_{n} <
  \frac{\epsilon^{\prime\prime}}{2}$ e $\sup_{0 \leq s \leq T}
  |\lambda_n^\infty(s) - s| < \frac{\epsilon^{\prime\prime}}{2}$, teremos
  que, para todo $y\in\Nzb$ e $s \leq \delta_n$:
  \begin{gather*}
    |\lambda_n^y(t) - s - t| \leq 
     |\lambda_n^y(t) - t | + |s| \leq
     \sup_{r \in [0, T]} |\lambda_n^\infty(r) - r | + |s|
     < \epsilon^{\prime\prime}\\
    |t - s - t |  = |s| < \epsilon^{\prime\prime}.
  \end{gather*}
  
  Portanto temos que:
  \begin{displaymath}
    d(X^\infty(\lambda_n^y(t)-s), X^\infty(t-s)) \leq
     d(X^\infty(\lambda_n^y(t)-s), X^\infty(t))+
     d(X^\infty(t), X^\infty(t-s))
     < \epsilon^{\prime}.
  \end{displaymath}

  De onde finalmente concluímos que quase certamente para todo $n > n_0$:
  \begin{displaymath} 
    \sup_{\stackrel{y > k_n}{y \in A}}
    \sup_{0 \leq s \leq \delta_n} \left\lvert
      f(X^\infty(\lambda_n^y(t)-s)) -
      f(X^\infty(t-s))
    \right\rvert < \epsilon .
  \end{displaymath}

  Como $\epsilon$ foi tomado arbitrariamente, concluímos que essa
  quantidade vai a zero \qc.
\end{proof}

%% ------------------------------------------------------------------------- %%

\section{Propriedade Forte de Markov}
\label{sec:prop-forte-markov}

Um resultado clássico - ver, por exemplo, \cite{fristedt:97} - diz que
um processo de Feller markoviano é fortemente markoviano. Já
comentamos que o Processo K não homogêneo pode não ser de Feller, mas
ele goza de algumas propriedades que irão possibilitar demonstrar a
propriedade forte de Markov de forma análoga à maneira pela qual a
demonstramos para processos de Feller.

Para cada $t > 0$, vamos denotar por $\FF_t$ a $\sigma$-álgebra gerada
por $\{X(s): s \in [0, t]\}$ e por $\FF$ a $\sigma$-álgebra do espaço
de probabilidades onde estamos trabalhando.

\begin{teorema}
  \label{teo:markov-forte}
  O Processo K é fortemente Markoviano.
\end{teorema}
\begin{proof}
  Fixada uma $f: \Nzb \to \R$ contínua e $\tau$ um tempo de parada
  finito para $(\FF_t)_{t \geq 0}$ . Vamos mostrar que quase
  certamente:
  \begin{equation}
    \label{eq:markov-forte}
    \E \left[ f(X(\tau + t) \middle| \FF_\tau \right]
    = \Psi_t f (X(\tau)),
  \end{equation}
  onde $\FF_\tau$ é definida da seguinte forma:
  \begin{displaymath}
    \FF_\tau = \left\{
      A \in \FF:  A \cap \{\tau \leq t\} \in \FF_t \quad \forall t \geq 0 
    \right\}
  \end{displaymath}


  Para cada $h > 0$, denotemos por: 
  \begin{displaymath}
    \tau_h := \inf\left\{
      t \geq \tau: t = k h, k \in \N
    \right\}.
  \end{displaymath}

  Observemos que $\tau_h \leq t$ se e somente se $\tau \leq \lfloor t
  \rfloor_h$, onde $\lfloor t \rfloor_h := \max\{k h \leq t: k \in
  \N\}$ Assim $\{\tau_h \leq k h\} \in \FF_{\lfloor t \rfloor_h}
  \subseteq \FF_t$ e portanto $\tau_h$ é um tempo de parada para a
  filtração $(\FF_{t})_{t \geq 0}$. Notemos ainda que $\tau \leq \tau_h
  < \tau + h$. Assim $\tau_h$ é \qc finito e $\tau_h \to \tau$ \qc
  quando $h \to 0$.

  Como em tempo discreto a propriedade de Markov é equivalente à
  propriedade forte de Markov, então teremos que para todo $h > 0$ e
  $k \in \N$ vale que:
 \begin{displaymath}
    \E \left[ f(X(\tau_h + k h)) \middle| \FF_{\tau_h} \right]
    = \Psi_{k h} f (X(\tau_h)).
  \end{displaymath}
  
  Fixemos um $t > 0$ e tomemos $h = t/k$, reescrevendo a expressão acima
  teremos que:
  \begin{equation}
    \label{eq:markov-forte-esp-cond}
    \E \left[ f(X(\tau_h + t)) \middle| \FF_{\tau_h} \right]
    = \Psi_t f (X(\tau_h)).
  \end{equation}
  

  Vamos mostrar agora que para todo $h > 0$, $\FF_{\tau} \subseteq
  \FF_{\tau_h}$. Para isso tomemos um $A \in \FF_\tau$ e um $t \geq 0$
  arbitrários. Como $\{\tau \leq t\} \subseteq \{\tau_h \leq t\}$,
  então:
  \begin{displaymath}
    A \cap \{ \tau_h \leq t \} =
    A \cap \{ \tau \leq t \} \cap \{ \tau_h \leq t \}.
  \end{displaymath}
  
  Observemos agora que $A \cap \{ \tau \leq t \} \in \FF_t$ porque $A
  \in \FF_\tau$, enquanto que $ \{ \tau_h \leq t \} \in \FF_t$ porque
  $\tau_h$ é tempo de parada. Assim concluímos que $A \cap \{ \tau_h
  \leq t \} \in \FF_t$ e, como isso vale para todo $t$, então $A \in
  \FF_{\tau_h}$.

  Usando a definição de esperança condicional em
  \eqref{eq:markov-forte-esp-cond}, teremos que para todo evento $B
  \in \FF_{\tau} \subseteq \FF_{\tau_h}$:
  \begin{equation}
    \label{eq:markov-forte-discreto}
    \E \left[ f(X(\tau_h + t)) \ind\{B\} \right]
    = \E \left[ \Psi_t f (X(\tau_h)) \ind\{B\} \right].
  \end{equation}

  Agora iremos fazer $h \searrow 0$ através de ``divisores'' de $t$.

  Como $f$ é uma função contínua e como $X$ é um processo contínuo à
  direita e $\tau_h \searrow \tau$ quando $h \searrow 0$, então
  $f(X(\tau_h + t)) \to f(X(\tau + t))$ \qc, e como $f$ é limitada,
  então a parte da esquerda de \eqref{eq:markov-forte-discreto}
  converge para:
  \begin{displaymath}
    \E \left[ f(X(\tau + t)) \ind\{B\} \right],
  \end{displaymath}
  quando $h \searrow 0$.

  Consideremos que tenhamos mostrado que $\Psi_t f (X(\tau_h)) \to
  \Psi_t f(X(\tau))$ \qc quando $h \searrow 0$. Com isso, já que
  $\Psi_t f$ é limitada, teremos que a parte da direita de
  \eqref{eq:markov-forte-discreto} convergirá para:
  \begin{displaymath}
    \E \left[ \Psi_t f (X(\tau)) \ind\{B\} \right].
  \end{displaymath}

  Dessa forma, para todo $B \in \FF_{\tau}$, valerá que:
  \begin{displaymath}
    \E \left[ f(X(\tau + t)) \ind\{B\} \right]
    = \E \left[ \Psi_t f (X(\tau)) \ind\{B\} \right].
  \end{displaymath}

  Como $\Psi_t f(X(\tau))$ é mensurável em $\FF_\tau$, concluímos que:
  \begin{displaymath}
    \E \left[ f(X(\tau + t)) \middle| \FF_\tau \right]
    = \Psi_t f (X(\tau)),
  \end{displaymath}
  que é exatamente \eqref{eq:markov-forte}.

  Portanto só resta mostrar que:
  \begin{equation}
    \label{eq:convergencia-discretizacao-1}
    \Psi_t f (X(\tau_h)) \xrightarrow[\qc]{h \searrow 0}
    \Psi_t f(X(\tau)).
  \end{equation}
  
  É exatamente nesse ponto que nossa demonstração vai divergir da
  demonstração clássica. Se nosso processo fosse de Feller, então
  $\Psi_t f$ seria uma função contínua e essa convergência seria
  evidente. Para contornar esse problema, vamos usar a Proposição
  \ref{prop:gamma-somavel} de maneira parecida à que fizemos quando
  queríamos mostrar a propriedade de Markov simples.

  Vamos fixar uma realização do processo onde $\sum_{x \in V_t}
  \gamma_x < \infty$ para todo $t$, $\tau < \infty$ e $X(\tau_{h}) \to
  X(\tau)$ quando $h \to 0$. Tais realizações têm probabilidade 1.


  Tomemos uma sequência $h_n \searrow 0$ arbitrária. Vamos mostrar que:
  \begin{equation}
    \label{eq:convergencia-discretizacao-2}
    \Psi_t f (X(\tau_{h_n})) \xrightarrow[\qc]{n \to \infty}
    \Psi_t f(X(\tau)).
  \end{equation}

  Se isso valer para toda sequência $h_n$, então concluiremos
  \eqref{eq:convergencia-discretizacao-1}. Sem perda de generalidade
  podemos olhar apenas para as sequências onde $h_n \leq 1$ para todo
  $n$.

  Se existir um $n_0$ tal que $n > n_0$ implica que $X(\tau_{h_n}) =
  X(\tau)$, então \eqref{eq:convergencia-discretizacao-2} é evidente.

  Assim vamos supor o contrário, ou seja, que para todo $n_0$ existe
  um $n > n_0$ tal que $X(\tau_{h_n}) \neq X(\tau)$. Dessa forma
  concluímos que $X(\tau)$ não é um ponto isolado de $\Nzb$. Como
  $\infty$ é o único ponto que não é isolado na topologia de $\Nzb$,
  então $X(\tau) = \infty$.

  Vamos denotar por $y_n = X(\tau_{h_n})$. Desconsideraremos os
  índices $n$ onde $y_n = \infty$, já que neles a igualdade em
  \eqref{eq:convergencia-discretizacao-2} é óbvia.

  Como $h_n \leq 1$ e $\tau_h < \tau + h$, vale que $y_n \in V_{\tau +
    1}$ para todo $n$.  Assim concluímos que $\gamma_{y_n} \to 0$
  através da Proposição \ref{prop:gamma-somavel} já que $y_n \to
  \infty$.

  Se tomarmos um $n$ grande o suficiente, de forma que $\gamma_{y_n}
  T_0 < t$, usando a Proposição \ref{prop:reinicia-infinito}, teremos
  que:
  \begin{align*}
    f(X^{\infty}(t)) - f(X^{y_n}(t)) &=
    f(X^{\infty}(t)) - f(X^{\infty}(t-\gamma_{y_n} T_0)).
  \end{align*}

  Esta quantidade vai a zero \qc porque $\gamma_{y_n} T_0 \to 0$
  \qc quando $n \to \infty$, $t$ é \qc um ponto de continuidade do
  processo e $f$ é uma função contínua.

  Como $f$ é limitada, utilizando o Teorema da Convergência Dominada,
  concluíremos que:
  \begin{align*}
    \Psi_t f(\infty) - \Psi_t f (y_n)
    &= \E \left[
      f(X^{\infty}(t)) - f(X^{y_n}(t))
    \right] \xrightarrow{n \to \infty} 0.
    \qedhere
  \end{align*}
\end{proof}



%%% Local Variables: 
%%% TeX-master: "tese"
%%% End: 
