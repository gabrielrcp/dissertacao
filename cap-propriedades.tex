%% ------------------------------------------------------------------------- %%
\chapter{Propriedades Básicas do Processo K}
\label{cap:propriedades}

Nesse capítulo vamos mostrar uma maneira de aproximar o processo K
por processos Markovianos de salto. Isso, além de nos fornecer uma
maneira intuitiva de encarar o processo, vai nos servir de ferramenta
para mostrar que o processo é Càdlàg e Markoviano.

No final do capítulo ainda enunciaremos um resultado que tratará do
quanto tempo o processo passa no infinito.

%% ------------------------------------------------------------------------- %%

\section{Aproximações}
\label{sec:aproximacoes}

Fixado um natural $n \in \Nz$, vamos ``truncar'' nosso processo em
$\{1, 2, \ldots, n, \infty\}$. Depois vamos mostrar mostrar que esses
processos truncados convergem ao processo original.

Primeiramente, para $n \in \Nz$ e $y \in \{1, \ldots, n, \infty\}$,
vamos definir:
\begin{equation}
  \Gamma^y_n (t) = \gamma_y T_0^y
  + \sum_{x =1}^{n} \sum_{i = 1}^{N_x(t)}
  \gamma_x T_i^x
  + ct.
\end{equation}

O processo truncado em $n \in \Nz$, com estado inicial $y \in \{1,
\ldots, n, \infty\}$ será:
\begin{equation}
  X^y_n(t) = \begin{cases}
    y, & \textrm{ se }  t < \gamma_y T_0^y\\
    x, & \textrm{ se } \Gamma^y_n(\sigma_i^x-) \leq t <
    \Gamma^y_n(\sigma^x_i)
    \textrm{ para algum } i \\
    \infty, & \textrm{ caso contrário.}
  \end{cases}
\end{equation}

\begin{proposicao}
  O processo $X_n^y$ é Càdlàd e Markoviano.
\end{proposicao}
\begin{proof}
  Como visto na seção \ref{sec:visualizacao}, o processo $X_n^y$ é um
  processo Markoviano de saltos. Além disso ele é Càdlàg por
  construção.
\end{proof}

Note que nós construímos $(X_n^y)_{n \in \Nz} e x^y$ num mesmo espaço
de probabilidade, de forma que $X_n^y$ sejam aproximações de $X^y$.
Assim é natural perguntar se $X_n^y$ converge para $X^y$.

\begin{teorema}
  \label{teo:convergencia}
  Se considerarmos a topologia introduzida na seção
  \ref{sec:topologia}, então $X_n^y \xrightarrow{n \to \infty} X^y$
  \qc na métrica de Skorohod.
\end{teorema}

A métrica de Skorohod é uma métrica sobre o espaço das tragetórias que
permite pequenas distorções temporais. Ele é uma métrica muito usada
quando estamos trabalhando com tragetórias Càdlàg.  Como referência
recomendamos \cite{billingsley:99} e \cite{ethier:86}.

\begin{proof}
  Usando o teorema xyz de \cite{ethier:86}, vamos mostrar... 
\end{proof}

\begin{corolario}
  \label{cor:proc_cadlag}
  $X^y$ é \qc càdlàg.
\end{corolario}

\begin{proof}
  A métrica de Skorohod é completa em $D$, ...
\end{proof}


%% ------------------------------------------------------------------------- %%

\section{Propriedade de Markov}

\begin{teorema}
  \label{teo:proc_markov}
  $X^y$ é um processo markoviano.
\end{teorema}

\begin{proof}
  ....
\end{proof}

\begin{proposicao}
  \label{prop:proc_feller}
  $X^y$ é um processo de Feller.
\end{proposicao}
\begin{proof}
  Um processo é de Feller se o seu semigrupo de transição for um
  operador contínuo...
\end{proof}



\begin{corolario}
  \label{cor:proc_fort_markov}
  $X^y$ é fortemente markoviano.
\end{corolario}
\begin{proof}
  Um processo de Feller markoviano é fortemente markoviano. Ver (???)
\end{proof}




%%% Local Variables: 
%%% TeX-master: "tese"
%%% End: 
