%% ------------------------------------------------------------------------- %%
\chapter{Conclusões}
\label{cap:conclusao}

Tentamos nessa dissertação fazer um estudo razoavelmente completo
sobre os processos K não homogênenos. Usamos de uma abordagem
construtiva e, na medida da nossa capacidade, mantívemos nossos
argumentos em nível elementar, não abusando de teorias sofisticadas.

Como sugestões para estudos futuros, podemos citar:

\begin{itemize}
\item Encontrar o gerador no caso não homogêneo. Além das dificuldades
  óbvias em generalizar nossos argumentos, o fato do processo
  poder não ser de Feller introduz uma dificuldade adicional  de não
  podermos enxergar o semigrupo de transição como um operador sobre as
  funções contínuas de $\Nzb$.

\item Encontrar uma fórmula mais específica para a dimensão de
  Hausdorff do tempo em que o processo passa no infinito. Nós
  apenas apresentamos cotas para \eqref{eq:dim-hausdorff}.
\end{itemize}



%%% Local Variables: 
%%% TeX-master: "tese"
%%% End: 
