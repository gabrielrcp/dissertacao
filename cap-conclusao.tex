%% ------------------------------------------------------------------------- %%
\chapter{Conclusões}
\label{cap:conclusao}

Nessa dissertação fizemos um estudo razoavelmente completo sobre os
Processos K não homogênenos. Usamos uma abordagem construtiva e
buscamos ao máximo mater nossos argumentos em nível mais elementar,
sem abusar de teorias sofisticadas.

Para estudos futuros, podemos sugerir:

\begin{itemize}
\item Calcular o gerador no caso não homogêneo. Além das dificuldades
  óbvias em generalizar nossos argumentos, o fato do processo poder
  não ser de Feller introduz uma dificuldade adicional de não podermos
  enxergar o semigrupo de transição como um operador sobre as funções
  contínuas de $\Nzb$.

\item Encontrar uma fórmula mais específica para a dimensão de
  Hausdorff do tempo em que o processo passa no infinito. Nós
  apenas apresentamos cotas para \eqref{eq:dim-hausdorff}.
\end{itemize}



%%% Local Variables: 
%%% TeX-master: "tese"
%%% End: 
