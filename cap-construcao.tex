%% ------------------------------------------------------------------------- %%
\chapter{Construção do processo}
\label{cap:construcao}

%% ------------------------------------------------------------------------- %%
\section{Definições}
\label{sec:definicoes}

Chamaremos de $\N$ o conjunto dos número naturais. Por convenção
assumiremos que o zero não está nesse conjunto, isso é, $\N = \{ 1, 2,
\ldots \}$. A esse conjunto adicionaremos um novo símbolo, que
denotaremos por $\infty$. Chamaremos esse novo conjunto de $\Nb$.

Muniremos $\Nb$ da seguinte métrica:
\begin{equation}
  \label{eq:metrica}
  d(x, y) = \left\lvert \frac{1}{x} - \frac{1}{y} \right\rvert,
\end{equation}
sob a convenção de que $\frac{1}{\infty} = 0$. Com essa métrica,
$\Nb$ é um espaço compacto. 

O processo K será um processo estocástico em tempo contínuo, que terá
como suporte o conjunto $\Nb$. Chamaremos os elementos de $\Nb$ de
estados ou sítios do processo.

Como parâmetros do processo, teremos duas funções,
 $\gamma: \Nb \to [0, \infty)$ e $\lambda: \Nb \to [0, \infty)$,
além de uma constante $c \in [0, \infty)$.
Para um estado $x \in \Nb$, podemos interpretar $\gamma(x)$ como
o tempo médio entre um processo entrar e sair de $x$ e $lambda(x)$
vai controlar a taxa com que o processo tende a entrar em
$x$. Enquanto que a constante $c$ controla o tempo em que o estado
fica em $\infty$.

Seguindo \cite{fontes:08}, iremos definir o processo K através de uma
construção. O espaço de probabilidade onde iremos definí-lo será um
espaço que admita as sequintes famílias de variáveis aleatórias
idependentes:

\begin{itemize}
  \item $\{ N_x: x \in \N \}$: processos pontuais de Poisson em de taxa
    $1$.
  \item $\{T_n^x: x \in \N , \, n = 0, 1, 2, 3, \ldots \}$:
    variáveis aleatórias exponencias de média $1$.
\end{itemize}

Para $t \geq 0$, iremos denotar por $N_x(t)$ o número de marcas do
processo de poisson  $N_x$ no intervalo $[0, t]$, marcas essas que
iremos denotar por $0 < \sigma_1^x < \sigma_2^x < \ldots$ .

Agora podemos definir uma função aleatória que será nossa principal
ferramenta para trabalhar com esse processo:

\begin{equation}
  \Gamma_y (t) = \gamma(y) T_0^y
                 + \sum_{x \in \N} \sum_{n = 1}^{N_x( \lambda (x) t)}
                       \gamma(x) T_n^x
                 + ct
  \qquad y \in \Nb, \quad t \geq 0.
\end{equation}

Chamaremos o processo K iniciado em $y$ de $X^y$, ele é definido
da seguinte forma:

\begin{equation}
  X^y (t) =
  \begin{cases}
    y & \textrm{ se }  t \leq \gamma(y) \\
    x & \textrm{ se }  \Gamma(t-) \leq \sigma^x_i < \Gamma(t)
       \textrm{ para algum } i \\
    \infty & \textrm{ caso contrário.}
  \end{cases}
\end{equation}


%%% Local Variables: 
%%% TeX-master: "tese"
%%% End: 
