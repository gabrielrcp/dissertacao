%% ------------------------------------------------------------------------- %%
\chapter{Construção do processo}
\label{cap:construcao}

%% ------------------------------------------------------------------------- %%
\section{Definições}
\label{sec:definicoes}

Chamaremos de $\N$ o conjunto dos número naturais. Por convenção
assumiremos que o zero não está nesse conjunto, isso é, $\N = \{ 1, 2,
\ldots \}$. A esse conjunto adicionaremos um novo símbolo, que
denotaremos por $\infty$. Chamaremos esse novo conjunto de $\Nb$.

Muniremos $\Nb$ da seguinte métrica:
\begin{equation}
  \label{eq:metrica}
  d(x, y) = \left\lvert \frac{1}{x} - \frac{1}{y} \right\rvert,
\end{equation}
sob a convenção de que $\frac{1}{\infty} = 0$. Com essa métrica $\Nb$
é um espaço compacto.

O processo K será um processo estocástico em tempo contínuo, cujo
espaço de estados será $\Nb$. Chamaremos os elementos de $\Nb$ de
estados ou sítios do processo.

Como parâmetros do processo, teremos duas famílias de números reais
estritamente positivos, $\gamma_x$ e $\lambda_x$, $x \in \N$. Além de
uma constante $c \in [0, \infty)$.  Para um estado $x \in \Nb$,
podemos interpretar $\gamma_x$ como o tempo médio para o processo sair
de $x$ e $\lambda_x$ vai controlar a taxa com que o processo tende a
entrar em $x$. Enquanto que $c$ controla o tempo em que o estado fica
em $\infty$.

Seguindo \cite{fontes:08}, iremos definir o processo K através de uma
construção. O espaço de probabilidade, $\Omega$, onde iremos defini-lo
será um espaço que admita as seguintes famílias de variáveis
aleatórias independentes:

\begin{itemize}
\item $\{ N_x: x \in \N \}$: processos pontuais de Poisson
  independentes, onde $N_x$ tem taxa $\lambda_x$ para cada $x \in \N$.
\item $\{T_n^x: x \in \Nb , \, n = 0, 1, 2, 3, \ldots \}$: variáveis
  aleatórias exponencias de média $1$.
\end{itemize}

Para $t \geq 0$, iremos denotar por $N_x(t)$ o número de marcas do
processo de poisson $N_x$ no intervalo $[0, t]$, marcas essas que
iremos denotar por $0 < \sigma_1^x < \sigma_2^x < \ldots$ .

Agora podemos definir uma função aleatória que será nossa principal
ferramenta para trabalhar com esse processo. Para $t \geq 0$, sob a
convenção que $\gamma_\infty = 0$:

\begin{equation}
  \label{def:Gamma}
  \Gamma_y (t) = \gamma_y T_0^y
  + \sum_{x \in \N} \sum_{n = 1}^{N_x(t)}
  \gamma_x T_n^x
  + ct
\end{equation}

Para que essa função seja \qc finita, vamos impor a restrição de que
\begin{equation}
  \sum_{x \in \N} \lambda_x\gamma_x < +\infty
\end{equation}

Chamaremos o processo K iniciado em $y \in \Nb$ de $X^y$. Ele é
definido, para $t \geq 0$, da seguinte forma:

\begin{equation}
  \label{def:procK}
  X^y (t) =
  \begin{cases}
    y & \textrm{ se }  t < \gamma_y T_0^y\\
    x & \textrm{ se } \Gamma_y(\sigma_i^x-) \leq t <
    \Gamma_y(\sigma^x_i)
    \textrm{ para algum } i \\
    \infty & \textrm{ caso contrário.}
  \end{cases}
\end{equation}

Note que para que essa definição faça sentido, temos que o limite a
esquerda $\Gamma_y (\sigma_i^x-)$ deve existir. Isso será estabelecido
na próxima proposição.

\begin{proposicao}
  \label{prop:gamma-cadlag}
  A função $\Gamma_y$ é quase certamente càdlàg. Isso é, é contínua à
  direita e tem limites à esquerda.
\end{proposicao}
\begin{proof}

  Consideremos $A$ o conjunto de todas as trajetórias $\omega \in
  \Omega$ tais que $\Gamma_y(T)$ seja finito, para todo $T$ positivo e
  racional, além disso não haja marcas repetidas na justaposição dos
  processos de Poisson $\{N_x: x \in \N\}$.


 Fixe $T > 0$, e vamos definir o evento:
  \begin{displaymath}
    A_T = \left\{
      \omega \in \Omega: \Gamma_y(T) < \infty
    \right\}.
  \end{displaymath}
  E tomemos $A = \cup_{T=1}^{\infty} A_T$. Consideremos ainda $B$ o
  evento que diz que não há marcas repetidas na justaposição dos
  processos de Poisson $N_x$.

  Como $P(A_T) = 1 = P(B)$, seque que $P(A \cap B) = 1$. Assim vamos
  mostrar que toda trajetória $\omega \in A \cap B$ é càdlàg.

  Se tomarmos $t \geq s \geq 0$, temos que:
  \begin{equation*}
    \Gamma_y(t) - \Gamma_y(s) = 
    c (t - s) + 
    \sum_{x \in \N} \sum_{n = N_x(s)+1}^{N_x(t)} \gamma_x T^x_n
    \geq 0 .
  \end{equation*}


  Assim $\Gamma_y$ é não decrescente. Tomando um $T > t$, temos ainda
  que $0 \leq \Gamma_y(s) \leq \Gamma_y(t) \leq \Gamma_y(T) <
  \infty$. Dessa maneira estamos lidando com sequências monótonas
  limitadas, o que estabelece a existência dos limites à direita e à
  esquerda.



  Para mostrar à continuidade à direita, tome $t \geq 0$ e uma
  sequência $(t_n)$ que convirja pela direita para $t$.
  \begin{equation*}
    \E(|\Gamma_y(t_n) - \Gamma_y(t)|) =
    (t_n - t) \left(
      c + 
      \sum_{x \in \N}\lambda_x \gamma_x 
    \right) \xrightarrow{n \to \infty} 0 
  \end{equation*}
  Assim $\Gamma_y(t_n)$ converge na norma $L_1$ para $\Gamma_y(t)$ e
  portanto também converge em probabilidade. Assim, como existe o
  limite à direita \qc, então esse limite deve ser igual à
  $\Gamma_y(t)$.


\end{proof}


\section{Visualização}



\section{Restrições}



%%% Local Variables: 
%%% TeX-master: "tese"
%%% End: 
