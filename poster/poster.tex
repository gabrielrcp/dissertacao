% We switch to portrait mode. This works as advertised.
\documentclass[a1,portrait]{a0poster}
% You might find the 'draft' option to a0 poster useful if you have
% lots of graphics, because they can take some time to process and
% display. (\documentclass[a0,draft]{a0poster})

\usepackage[utf8]{inputenc}
\usepackage[T1]{fontenc}
\usepackage[english]{babel}
%\usepackage{multirow}
%\usepackage{colortbl}
\usepackage{amsmath}
\usepackage{amssymb}
\usepackage[abbrev]{amsrefs}
\usepackage{amsthm}

\usepackage[fixlanguage]{babelbib}
%\usepackage[round,sort]{natbib}
%\usepackage{hyperref}


\pagestyle{empty}
\setcounter{secnumdepth}{0}


% The textpos package is necessary to position textblocks at arbitary 
% places on the page.
\usepackage[absolute]{textpos}

\usepackage{graphics,wrapfig,times}
\usepackage[pdftex]{graphicx}
\usepackage[usenames,svgnames,dvipsnames]{xcolor}
\usepackage{tikz}

\newtheorem*{teorema}{Theorem}%[section]
\newtheorem*{lema}{Lemma}%[section]
\newtheorem*{proposicao}{Proposicion}%[section]
\newtheorem*{definicao}{Definition}%[section]
\newtheorem{problema}{Problem}

\newcommand{\Prob}{\mathbb{P}}
\newcommand{\E}{\mathbb{E}}
\newcommand{\Xb}{\bar{X}}
\newcommand{\R}{\mathbb{R}}
\newcommand{\Z}{\mathbb{Z}}
\newcommand{\N}{\mathbb{N}}

\newcommand{\ind}{\mathbb{I}}
\newcommand{\qc}{\emph{q.c.}}

\newcommand{\FF}{\mathcal{F}}
\newcommand{\GG}{\mathcal{G}}


\def\Title#1{\noindent{\Huge\color{DarkBlue} #1}} %titulos
\def\Subhead#1{\noindent{\large #1}} %autores


\TPGrid[20mm,10mm]{31}{50}  % 3 - 1 - 7 - 1 - 3 Columns

\parindent=0pt
%\parindent=1cm
\parskip=0.5\baselineskip



\begin{document}

\begin{textblock}{25}(4,1)
\baselineskip=3\baselineskip \Title{
  A study on the non-homogeneous K-Process
}
\end{textblock}

\begin{textblock}{3}(0,0)
  \includegraphics[scale = 0.35]{logo-ime.jpg}
\end{textblock}

\begin{textblock}{2}(28,0)
  \includegraphics[scale = 0.5]{logo-numec.jpg}
\end{textblock}

\begin{textblock}{10}(3,2)
  \Subhead{
    \begin{center}
      Gabriel Ribeiro da Cruz Peixoto \\
      \emph{IME-USP}
    \end{center}
  }
\end{textblock}

\begin{textblock}{10}(13,2)
  \Subhead{
    \begin{center}
      Advisor: Luiz Renato Gonçalves Fontes \\
      \emph{IME-USP}
    \end{center}
  }
\end{textblock}




\begin{textblock}{15}(0,4)

\section{Introduction}

\nocite{fontes:08}

\end{textblock}


\begin{textblock}{15}(16,4) 

\section{Some Results}

\bibliographystyle{plainnat}
\bibliography{../bibliografia}

Support: CNPQ


\end{textblock}

\end{document}
