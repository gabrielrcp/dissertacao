% We switch to portrait mode. This works as advertised.
\documentclass[a1,portrait]{a0poster}
% You might find the 'draft' option to a0 poster useful if you have
% lots of graphics, because they can take some time to process and
% display. (\documentclass[a0,draft]{a0poster})

\usepackage[utf8]{inputenc}
\usepackage[T1]{fontenc}
\usepackage[english]{babel}
%\usepackage{multirow}
%\usepackage{colortbl}

\usepackage{amsmath}
\usepackage{amssymb}
\usepackage[abbrev]{amsrefs}
\usepackage{amsthm}
\usepackage{mathrsfs}
\usepackage{yfonts}

\usepackage[fixlanguage]{babelbib}
%\usepackage[round,sort]{natbib}
%\usepackage{hyperref}


\pagestyle{empty}
\setcounter{secnumdepth}{0}


% The textpos package is necessary to position textblocks at arbitary 
% places on the page.
\usepackage[absolute]{textpos}

\usepackage{graphics,wrapfig,times}
\usepackage[pdftex]{graphicx}
\usepackage[usenames,svgnames,dvipsnames]{xcolor}
\usepackage{tikz}

\newtheorem{teorema}{Theorem}
\newtheorem{lema}{Lemma}
\newtheorem{proposicao}{Proposition}
\newtheorem{definicao}{Definition}


\newcommand{\AAA}{{\mathcal{A}}}
\newcommand{\DDD}{{\mathfrak{D}}}
\newcommand{\FFF}{{\mathfrak{F}}}
\newcommand{\GGG}{{\mathfrak{G}}}

\newcommand{\CC}{{\mathcal{C}}}
\newcommand{\RR}{{\mathcal{R}}}
\newcommand{\II}{{\mathcal{I}}}
\newcommand{\RRb}{\overline{\mathcal{R}}}
\newcommand{\FF}{{\mathcal{F}}}
\newcommand{\PP}{{\mathcal{P}}}
\newcommand{\GG}{{\mathcal{G}}}


\newcommand{\N}{{\mathbb{N}}}
\newcommand{\Nb}{{\widebar{\N}}}
\newcommand{\Nz}{{\mathbb{N^*}}}
\newcommand{\Nzb}{{\mathbb{\overline{N}^*}}}
\newcommand{\Z}{{\mathbb{Z}}}
\newcommand{\R}{{\mathbb{R}}}
\newcommand{\E}{{\mathbb{E}}}


\newcommand{\qc}{{\emph{q.c.}} }
\newcommand{\ind}{{\mathbb{I}}}
\newcommand{\diam}{{\mathrm{diam}}}


\def\Title#1{\noindent{\Huge\color{DarkBlue} #1}} %titulos
\def\Subhead#1{\noindent{\large #1}} %autores


\TPGrid[20mm,10mm]{31}{50}  % 3 - 1 - 7 - 1 - 3 Columns

\parindent=0pt
%\parindent=1cm
\parskip=0.5\baselineskip



\begin{document}

\begin{textblock}{25}(4,1)
\baselineskip=3\baselineskip \Title{
  A study on the non-homogeneous K-Process
}
\end{textblock}

\begin{textblock}{3}(0,0)
  \includegraphics[scale = 0.35]{logo-ime.jpg}
\end{textblock}

\begin{textblock}{2}(28,0)
  \includegraphics[scale = 0.5]{logo-numec.jpg}
\end{textblock}

\begin{textblock}{10}(3,2)
  \Subhead{
    \begin{center}
      Gabriel Ribeiro da Cruz Peixoto \\
      \emph{IME-USP}
    \end{center}
  }
\end{textblock}

\begin{textblock}{10}(13,2)
  \Subhead{
    \begin{center}
      Advisor: Luiz Renato Gonçalves Fontes \\
      \emph{IME-USP}
    \end{center}
  }
\end{textblock}


\begin{textblock}{15}(0,4)

  \section{Construction}

  We consider a continuous time processes over a countable state space
  \begin{displaymath}
    \Nzb = \{1, 2, \ldots \} \cup \{ \infty \}.
  \end{displaymath}

\begin{itemize}
\item $\{ N_x: x \in \Nz \}$: processos pontuais de Poisson
  independentes, onde $N_x$ tem taxa $\lambda_x$ para cada $x \in \Nz$.
\item $\{T_0\} \cup \{T_n^x: x \in \Nz , \, n = 1, 2, 3, \ldots \}$:
  variáveis aleatórias exponenciais de média $1$.
\end{itemize}


  \begin{equation}
  \label{def:Gamma}
  \Gamma(t) := \Gamma^y_c (t) = \gamma_y T_0
  + \sum_{x \in \N} \sum_{n = 1}^{N_x(t)}
  \gamma_x T_n^x
  + ct
\end{equation}

Vamos impor algumas restrições sobre nossos parâmetros:
\begin{align}
  \label{eq:rest-gamma-lambda}
  \sum_{x \in \Nz} \lambda_x\gamma_x < +\infty\\
  \label{eq:rest-lambda}
  \sum_{x \in \Nz} \lambda_x = \infty
\end{align}


\begin{equation}
  \label{def:procK}
  X(t) = X^y_c (t) =
  \begin{cases}
    y, & \textrm{ se }  t < \gamma_y T_0\\
    x, & \textrm{ se } \Gamma^y_c(\sigma_i^x-) \leq t <
    \Gamma^y_c(\sigma^x_i)
    \textrm{ para algum } i \\
    \infty, & \textrm{ caso contrário.}
  \end{cases}
\end{equation}

\begin{teorema}
  This construction wields a strongly markovian càdlàg process.
\end{teorema}


\section{Aproximations}

For $n \in \Nz$ e $y \in \{1, \ldots, n, \infty\}$,
vamos definir:
\begin{displaymath}
  \Gamma^{(n)} (t) := \Gamma^{y,(n)}_c (t) = \gamma_y T_0
  + \sum_{x =1}^{n} \sum_{i = 1}^{N_x(t)}
  \gamma_x T_i^x
  + ct.
\end{displaymath}

O processo truncado em $n \in \Nz$, com estado inicial $y \in \{1,
\ldots, n, \infty\}$ será:
\begin{displaymath}
  X^{(n)}(t) = X^{y,(n)}_c(t) = \begin{cases}
    y, & \textrm{ se }  t < \gamma_y T_0 \\
    x, & \textrm{ se } \Gamma^{y,(n)}_c(\sigma_i^x-) \leq t <
    \Gamma^{y,(n)}_c(\sigma^x_i)
    \text{ para algum } i \\
    \infty, & \textrm{ caso contrário.}
  \end{cases}
\end{displaymath}


\begin{teorema}
  The truncated process converges almost surelly to the K-Process on
  the Skorohord norm.
\end{teorema}


  \section{Transition rates}

  The Q-matrix of a Process is defined by:
  \begin{displaymath}
    Q = \lim_{t \searrow 0} \frac{P(t) - I}{t}.
  \end{displaymath}

  \begin{teorema}
    When $c > 0$, the Q-matrix of the K-Process is given by:
    \begin{displaymath}
      Q = \left(
        \begin{array}{ccccc}
          -\frac{1}{\gamma_1} & 0 & 0 & \cdots & \frac{1}{\gamma_1}\\
          0 & -\frac{1}{\gamma_2} & 0 & \cdots & \frac{1}{\gamma_2}\\
          0 & 0 & -\frac{1}{\gamma_3} & \cdots & \frac{1}{\gamma_3}\\
          \vdots & \vdots & \vdots & \ddots & \vdots \\
          \frac{\lambda_1}{c} & \frac{\lambda_2}{c} &
          \frac{\lambda_3}{c} & \cdots & -\infty\\
        \end{array}
      \right).
    \end{displaymath}
  \end{teorema}

  \begin{teorema}
    When $c = 0$, the Q-matrix of the K-Process is given by:
    \begin{displaymath}
      Q = \left(
        \begin{array}{ccccc}
          -\frac{1}{\gamma_1} & 0 & 0 & \cdots & 0\\
          0 & -\frac{1}{\gamma_2} & 0 & \cdots & 0\\
          0 & 0 & -\frac{1}{\gamma_3} & \cdots & 0\\
          \vdots & \vdots & \vdots & \ddots & \vdots \\
          \infty & \infty & \infty & \cdots & -\infty\\
        \end{array}
      \right).
    \end{displaymath}
  \end{teorema}
\end{textblock}


\begin{textblock}{15}(16,4) 
  

  \section{Invariant measure}

  \begin{teorema}
    The invariant measure of the K-Process is given by:
    \begin{equation}
      \label{eq:invariante}
      \pi(x) := \begin{cases}
        \frac{\lambda_x \gamma_x}{c + \sum_{y \in \Nz} \lambda_y \gamma_y}
        & \textrm{ se } x \in \Nz \\
        \frac{c}{c + \sum_{y \in \Nz} \lambda_y \gamma_y}
        & \textrm{ se } x = \infty \\
      \end{cases}
    \end{equation}
  \end{teorema}

  \section{Time spent on $\infty$}

  Consider:
  \begin{displaymath}
    \RR = \left\{ t \geq 0: X^\infty(t) = \infty \right\}
  \end{displaymath}

  Let $\dim_H(\RR)$ denote the Hausdorff dimension of this set.

\begin{teorema}
  \label{cor:log-haus}
  Suppose that $\inf_{x \in \Nz} \lambda_x > 0$ and $\sup_{x \in \Nz}
  \lambda_x < \infty$. Under these conditions:
  \begin{gather}
    \label{eq:log-haus-sup}
    \dim_H(\RR) \leq
    - \left( \limsup_{x \to \infty} \frac{\log \gamma_x}{\log x}
    \right)^{-1} \\
    \label{eq:log-haus-inf}
    \dim_H(\RR) \geq
    - \left( \liminf_{x \to \infty} \frac{\log \gamma_x}{\log x}
    \right)^{-1},
  \end{gather}
  under the conventions $-\frac{1}{-\infty} = 0$ and $-\frac{1}{0-} =
  +\infty$.
\end{teorema}


\section{Infinitesimal generator}

We only know the infinitesimal generator in the homogeneous case, that
is $\lambda_x = 1$ for every $x \in \Nz$.

In \cite{kendall:56} the infinitesimal generator $\AAA$ was calculated
in the case $c > 0$. It's domain are the set of function $f: \Nzb \to
\R$ such that $\lim_{x \to \infty} \frac{f(x) - f(\infty)}{\gamma_x} =
\sum_{x \in \Nz} [f(\infty) - f(x)]$. Over such functions:

\begin{align*}
  \AAA f (x) = \begin{cases}
    \displaystyle
    \frac{f(\infty) - f(x)}{\gamma_x} & \text{se } x \in \Nz\\
    \displaystyle
    \sum_{y\in \Nz} \frac{f(y) - f(\infty)}{c} & \text{se } x = \infty.
  \end{cases}
\end{align*}


\begin{teorema}
  When $c = 0$, if $f: \Nzb \to \R$ is a function such that: 
  \begin{gather}
    \sum_{x\in \Nz} |f(x)-f(\infty)| < \infty\\
    \sum_{x\in \Nz} \left( f(x)-f(\infty)\right) = 0\\
    \lim_{x \to \infty} \frac{f(x) - f(\infty)}{\gamma_x} \textrm{ exists},
  \end{gather}
  then:
  \begin{displaymath}
    \AAA f(x) = \begin{cases}
      \frac{f(\infty)- f(x)}{\gamma_x} & \text{ if } x \in \Nz \\
      \lim_{y \to \infty} \frac{f(\infty) - f(y)}{\gamma_y} & \text{
        if } x = \infty
    \end{cases}  
  \end{displaymath}
\end{teorema}

\nocite{fontes:08}
\nocite{gabriel:11}
\bibliographystyle{plainnat}
\bibliography{../bibliografia}

Support: CNPQ


\end{textblock}

\end{document}
