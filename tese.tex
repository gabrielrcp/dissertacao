% Arquivo LaTeX de exemplo de dissertação/tese a ser apresentados à CPG do IME-USP
% 
% Versão 3: Sat Feb 13 15:02:20 BRST 2010
%
% Criação: Jesús P. Mena-Chalco
% Revisão: Fabio Kon e Paulo Feofiloff
%  
% Obs: Leia previamente o texto do arquivo README.txt

\documentclass[11pt,twoside,a4paper]{book}

% ---------------------------------------------------------------------------- %
% Pacotes 
\usepackage[T1]{fontenc}
\usepackage[brazil]{babel}
\usepackage[utf8]{inputenc}
\usepackage[pdftex]{graphicx}           % usamos arquivos pdf/png como figuras
\usepackage{setspace}                   % espaçamento flexível
\usepackage{indentfirst}                % indentação do primeiro parágrafo
\usepackage{makeidx}                    % índice remissivo
\usepackage[nottoc]{tocbibind}          % acrescentamos a bibliografia/indice/conteudo no Table of Contents
\usepackage{courier}                    % usa o Adobe Courier no lugar de Computer Modern Typewriter
\usepackage{type1cm}                    % fontes realmente escaláveis
\usepackage{titletoc}
%\usepackage[bf,small,compact]{titlesec} % cabeçalhos dos títulos: menores e compactos
\usepackage[fixlanguage]{babelbib}
\usepackage[font=small,format=plain,labelfont=bf,up,textfont=it,up]{caption}
\usepackage[usenames,svgnames,dvipsnames]{xcolor}
\usepackage[a4paper,top=2.54cm,bottom=2.0cm,left=2.0cm,right=2.54cm]{geometry} % margens
%\usepackage[pdftex,plainpages=false,pdfpagelabels,pagebackref,colorlinks=true,citecolor=black,linkcolor=black,urlcolor=black,filecolor=black,bookmarksopen=true]{hyperref} % links em preto
\usepackage[pdftex,plainpages=false,pdfpagelabels,pagebackref,colorlinks=true,citecolor=DarkGreen,linkcolor=NavyBlue,urlcolor=DarkRed,filecolor=green,bookmarksopen=true]{hyperref} % links coloridos
\usepackage[all]{hypcap}                    % soluciona o problema com o hyperref e capitulos
\usepackage[round,sort,nonamebreak]{natbib} % citação bibliográfica textual(plainnat-ime.bst)
\fontsize{60}{62}\usefont{OT1}{cmr}{m}{n}{\selectfont}

% ---------------------------------------------------------------------------- %
% Cabeçalhos similares ao TAOCP de Donald E. Knuth
\usepackage{fancyhdr}
\pagestyle{fancy}
\fancyhf{}
\renewcommand{\chaptermark}[1]{\markboth{\MakeUppercase{#1}}{}}
\renewcommand{\sectionmark}[1]{\markright{\MakeUppercase{#1}}{}}
\renewcommand{\headrulewidth}{0pt}

% ---------------------------------------------------------------------------- %
\graphicspath{{./figuras/}}             % caminho das figuras (recomendável)
\frenchspacing                          % arruma o espaço: id est (i.e.) e exempli gratia (e.g.) 
\urlstyle{same}                         % URL com o mesmo estilo do texto e não mono-spaced
\makeindex                              % para o índice remissivo
\raggedbottom                           % para não permitir espaços extra no texto
\fontsize{60}{62}\usefont{OT1}{cmr}{m}{n}{\selectfont}
\cleardoublepage
\normalsize

\usepackage{amsmath}
\usepackage{amssymb}
\usepackage{amsthm}

\newcommand{\N}{{\mathbb{N}}}
\newcommand{\Nb}{{\bar{\N}}}
\newcommand{\R}{{\mathbb{R}}}
\newcommand{\qc}{\emph{q.c.}}
\newcommand{\E}{{\mathbb{E}}}

\newtheorem{teorema}{Teorema}[section]
\newtheorem{lema}[teorema]{Lema}
\newtheorem{proposicao}[teorema]{Proposição}
\newtheorem{corolario}[teorema]{Corolário}


% ---------------------------------------------------------------------------- %
% Corpo do texto
\begin{document}
\frontmatter 
% cabeçalho para as páginas das seções anteriores ao capítulo 1 (frontmatter)
\fancyhead[RO]{{\footnotesize\rightmark}\hspace{2em}\thepage}
\setcounter{tocdepth}{2}
\fancyhead[LE]{\thepage\hspace{2em}\footnotesize{\leftmark}}
\fancyhead[RE,LO]{}
\fancyhead[RO]{{\footnotesize\rightmark}\hspace{2em}\thepage}

\onehalfspacing  % espaçamento

% ---------------------------------------------------------------------------- %
% Capa
% Nota: O título para as teses/dissertações do IME-USP devem caber em um 
% orifício de 10,7cm de largura x 6,0cm de altura que há na capa fornecida pela SPG.
\thispagestyle{empty}
\begin{center}
  \vspace*{2.3cm}
  \textbf{\Large{Título do trabalho a ser apresentado à \\
      CPG para a dissertação/tese}}\\
  
  \vspace*{1.2cm}
  \Large{Gabriel Ribeiro da Cruz Peixoto}
  
  \vskip 2cm
  \textsc{
    Dissertação/Tese apresentada\\[-0.25cm] 
    ao\\[-0.25cm]
    Instituto de Matemática e Estatística\\[-0.25cm]
    da\\[-0.25cm]
    Universidade de São Paulo\\[-0.25cm]
    para\\[-0.25cm]
    obtenção do título\\[-0.25cm]
    de\\[-0.25cm]
    Mestre em Ciências}
  
  \vskip 1.5cm
  Programa: Nome do Programa\\
  Orientador: Prof. Dr. Luiz Renato Gonçalves Fontes

  \vskip 1cm
  \normalsize{Durante o desenvolvimento deste trabalho o autor recebeu auxílio
    financeiro da CNPq}
  
  \vskip 0.5cm
  \normalsize{São Paulo, fevereiro de 2010}
\end{center}

% ---------------------------------------------------------------------------- %
% Página de rosto (só para a versão final)
% Nota: O título para as teses/dissertações do IME-USP devem caber em um 
% orifício de 10,7cm de largura x 6,0cm de altura que há na capa fornecida pela SPG.
% \newpage
% \thispagestyle{empty}
%     \begin{center}
%         \vspace*{2.3 cm}
%         \textbf{\Large{Título do trabalho a ser apresentado à \\
%         CPG para a dissertação/tese}}\\
%         \vspace*{2 cm}
%     \end{center}

%     \vskip 2cm

%     \begin{flushright}
%     % Este exemplar corresponde à redação\\
%     % final da dissertação/tese devidamente corrigida\\
%     % e defendida por (Nome Completo do Aluno)\\
%     % e aprovada pela Comissão Julgadora.
%     %
% 	Esta versão definitiva da tese/dissertação\\
% 	contém as correções e alterações sugeridas pela\\
% 	Comissão Julgadora durante a defesa realizada\\
%     por (Nome Completo do Aluno) em 4/5/2010.

%     \vskip 2cm

%     \end{flushright}
%     \vskip 4.2cm

%     \begin{quote}
%     \noindent Comissão Julgadora:
    
%     \begin{itemize}
% 		\item Prof. Dr. Luiz Renato Gonçalves Fontes (orientador) - IME-USP
% 		\item Prof. Dr. Nome Completo - IME-USP [sem ponto final]
% 		\item Prof. Dr. Nome Completo - IMPA [sem ponto final]
%     \end{itemize}
      
%     \end{quote}
% \pagebreak

\pagenumbering{roman}     % começamos a numerar 

% ---------------------------------------------------------------------------- %
% Agradecimentos
\chapter*{Agradecimentos}
Agradecimentos ...

% ---------------------------------------------------------------------------- %
% Resumo
\chapter*{Resumo}
Resumo em português ...

\noindent \textbf{Palavras-chave:} palavra-chave1, palavra-chave2, palavra-chave3.

% ---------------------------------------------------------------------------- %
% Abstract
\chapter*{Abstract}
Abstract ...

\noindent \textbf{Keywords:} keyword1, keyword2, keyword3.

% ---------------------------------------------------------------------------- %
% Sumário
\tableofcontents    % imprime o sumário

% % ---------------------------------------------------------------------------- %
% \chapter{Lista de Abreviaturas}
% \begin{tabular}{ll}
%          CFT         & Transformada contínua de Fourier (\emph{Continuous Fourier Transform}).\\
%          DFT         & Transformada discreta de Fourier (\emph{Discrete Fourier Transform}).\\
%         EIIP         & Potencial de interação elétron-íon (\emph{Electron-Ion Interaction Potentials}).\\
%         STFT         & Tranformada de Fourier de tempo reduzido (\emph{Short-Time Fourier Transform}).\\
% \end{tabular}

% % ---------------------------------------------------------------------------- %
% \chapter{Lista de Símbolos}
% \begin{tabular}{ll}
%         $\omega$    & Frequência angular.\\
%         $\psi$      & Função de análise \emph{wavelet}.\\
%         $\Psi$      & Transformada de Fourier de $\psi$.\\
% \end{tabular}

% ---------------------------------------------------------------------------- %
% Listas de figuras e tabelas criadas automaticamente
\listoffigures            
\listoftables            

% ---------------------------------------------------------------------------- %
% Capítulos do trabalho
\mainmatter

% cabeçalho para as páginas de todos os capítulos
\fancyhead[RE,LO]{\thesection}

\singlespacing              % espaçamento simples
%\onehalfspacing            % espaçamento um e meio

\input cap-introducao        % associado ao arquivo: 'cap-introducao.tex'
\input cap-construcao        % associado ao arquivo: 'cap-introducao.tex'

% cabeçalho para os apêndices
\renewcommand{\chaptermark}[1]{\markboth{\MakeUppercase{\appendixname\ \thechapter}} {\MakeUppercase{#1}} }
\fancyhead[RE,LO]{}
\appendix


\chapter{Nome de um apêndice}
\label{ape:nome}

Texto texto texto texto texto texto texto texto texto texto texto texto texto
texto texto texto texto texto texto texto texto texto texto texto texto texto
texto texto texto texto texto texto.


\singlespacing

\renewcommand{\arraystretch}{0.85}
\captionsetup{margin=1.0cm}  % correção nas margens dos captions.

%%% Local Variables: 
%%% mode: latex
%%% TeX-master: "tese"
%%% End: 
      % associado ao arquivo: 'ape-conjuntos.tex'

% ---------------------------------------------------------------------------- %
% Bibliografia
\backmatter \singlespacing   % espaçamento simples
\bibliographystyle{plainnat-ime} % citação bibliográfica textual
\bibliography{bibliografia}  % associado ao arquivo: 'bibliografia.bib'

% ---------------------------------------------------------------------------- %
% Índice remissivo
% \index{TBP|see{periodicidade região codificante}}
% \index{DSP|see{processamento digital de sinais}}
% \index{STFT|see{transformada de Fourier de tempo reduzido}}
% \index{DFT|see{transformada discreta de Fourier}}
% \index{Fourier!transformada|see{transformada de Fourier}}
%
\printindex   % imprime o índice remissivo no documento 

\end{document}


%%% Local Variables: 
%%% mode: latex
%%% TeX-master: t
%%% End: 
